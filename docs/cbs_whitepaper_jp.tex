\documentclass[dvipdfmx,a4paper]{jsarticle}
\usepackage{amsmath,amssymb,multicol,indentfirst,tablefootnote}
\usepackage[dvipdfmx]{graphicx}

\def\tab{\hskip 2em}
\renewcommand\arraystretch{1.4}
\sloppy

\title{\textbf{Central Bank Stablecoin:\\日銀発行ステーブルコインの可能性}}
\author{原 健太朗\footnote{Email: xharaken@gmail.com. 本稿に示された見解は個人のものであり私の所属する組織のものではありません。}}
\date{2022 October}

\begin{document}

\maketitle

\begin{abstract}

本稿ではCentral Bank Stablecoin(CBS)を提案し、その可能性とリスクを検討する。CBSのアイディアは、中央銀行が法定通貨をパブリックブロックチェーン上でトークン化してステーブルコインを発行するというものである。本稿では日銀がCBSを発行した場合、以下の3つの可能性を持っていることを説明する。

\begin{itemize}
\item 既存のいかなるステーブルコインよりも信認のあるステーブルコインを作り出すことでブロックチェーン決済が拡大する。
\item 日銀の管理下にあるステーブルコインをブロックチェーンに流し込みそれがブロックチェーン決済の基軸通貨となることで、民間発行ステーブルコインの影響が弱まり世界金融不安定化のリスクが低減する。
\item 日本円がブロックチェーン決済の基軸通貨としての地位を得ることで、日銀がその貨幣発行益を手に入れるとともに日本円の国際競争力が高まる。
\end{itemize}

後述するように本提案は中央銀行デジタル通貨(CBDC)とは異なるものである。

\end{abstract}

\section{アイディア}

\textbf{Central Bank Stablecoin(CBS)}のアイディアはシンプルである。日銀はEthereumなどのパブリックブロックチェーン上にスマートコントラクトをデプロイして、ERC20トークン~\cite{erc20token}を発行する。発行ルールとしては、日銀は$X$円が預金された場合に$X$CBSコインを発行し、$X$CBSコインが返却された場合には$X$円との兌換を保証する。CBSは日本円をパブリックブロックチェーン上にトークン化したものであり実質的に日本円と同等の価値を持つ。

本稿ではCBSの可能性について概観したあと主要な論点について考察を深める。本稿は日銀がCBSを発行することを想定して議論を進めるが、信頼のある法定通貨を発行する先進国の中央銀行であればCBSを発行可能である。

\section{可能性}

\subsection{信認のあるステーブルコインの必要性}

標準的な貨幣理論によれば貨幣の基本的機能は「交換手段」「貯蔵手段」「価値尺度」の3つであり~\cite{davies2010history}、このうち貨幣が「貯蔵手段」「価値尺度」として機能するためには貨幣価値が安定していることが要求される。たとえばビットコインの対ドル価格は乱高下しており、「交換手段」としては機能するが「貯蔵手段」「価値尺度」としては機能しない。エルサルバドルではビットコインが法定通貨に指定されているが、その価格変動の大きさゆえに小売店はドルで価格表示を行っており(=「価値尺度」として機能していない)、そのときの時価でビットコインの支払いを受け入れる形を取っている(=「交換手段」としては機能している)。またビットコインの価格下落リスクから国民を守るために、手に入れたビットコインを手数料なしで即座にドルに換金するサービスを政府が提供している(=「貯蔵手段」として機能していない)。価値の安定しない貨幣は投機対象としては機能するが非投機的な実体経済の決済手段としては機能しない。

経済規模の小さい発展途上国は自国通貨を発行してもその価値を安定させることが難しいためドル化~\cite{calvo2002dollarization,selgin2005currency}やドルペッグを行うことが多いが、これは非投機的な実体経済の決済手段として価値の安定した貨幣が要求されるためである。逆に言えば、\textbf{価値の安定した貨幣が存在することが非投機的な実体経済を成長させるための必要条件}と言える。現在のブロックチェーンビジネスは投機的需要に支えられたものが大半を占めるが、信認のあるステーブルコインが存在すれば非投機的なブロックチェーン決済が促進されブロックチェーン経済圏が拡大することが期待できる。

なお既存の民間発行ステーブルコイン~\cite{arner2020stablecoins,moin2020sok}はいずれも十分な信認を得ているとは言いがたい。第一に、JPYCやUSDT~\cite{tether}などの法定通貨担保型ステーブルコインは流通しているコインに対して100\%の担保を保有することが要求されるが、たとえばUSDTを発行するTether社はUSDTに対して100\%担保を保有していないのではないかという疑惑がかけられている。また担保を保有する企業が倒産する可能性がありカウンタパーティリスクが存在する。第二に、MakerDAO~\cite{makerdao}などの暗号通貨担保型ステーブルコインは担保の暗号通貨の資産価値が暴落するリスクを抱えており、またガバナンストークンを利用したDAOで運用されるため将来どのようなルールでステーブルコインが発行管理されるかが予測できないリスクがある。第三に、Terra~\cite{terra}やEmpty Set Dollar~\cite{emptysetdollar}などのアルゴリズム型ステーブルコインは長期的に通貨価値を安定させることに失敗している~\footnote{著者の提案するJohnLawCoin~\cite{johnlawcoin}はアルゴリズム型ステーブルコインで長期的な通貨価値安定を目指すものである。}。これに対してCBSは日本円をパブリックブロックチェーン上にトークン化したものであるから実質的に日本円と等価であり、既存のいかなる民間発行ステーブルコインよりも信認され、非投機的なブロックチェーン決済の拡大に資すると考えられる。

\subsection{民間発行ステーブルコインがもたらす世界金融不安定化}

2020年12月にはEmpty Set Dollarが暴落、2022年5月にはTerraが暴落した。とくにTerraの暴落はアルゴリズム型ステーブルコインだけでなくステーブルコイン市場全体に波及し、Terraの共同創業者が資本市場法違反で起訴されたりFRBのパウエル議長がステーブルコイン規制の必要性を唱えるなど規制の流れが強まっている。ステーブルコインの価値は安定しているものだと市場は期待して保有している一方で、実際にはステーブルコインが暴落のリスクを抱えていることを考慮すれば、規制の議論が強まるのは当然と言える。今後ステーブルコインの市場規模が拡大すれば、2008年のリーマンショックが大規模な不良債権を発生させ世界金融危機を招いたのと同様に、ステーブルコインの暴落が世界金融を不安定化させるリスクが高まる。

ところが\textbf{法的規制は問題の本質的解決にならない}ことに留意したい。法定通貨担保型ステーブルコインの場合には担保を保有する主体としての法人が存在するため法によって規制することが可能であるが、暗号通貨担保型ステーブルコインやアルゴリズム型ステーブルコインはスマートコントラクトによって非中央集権的に実行されるため法で規制することができない。今後ステーブルコインへの市場需要が拡大すれば、これらの法的規制の効かない民間発行ステーブルコインの発行規模が拡大し、世界金融不安定化のリスクが拡大する懸念がある。

\textbf{民間発行ステーブルコインがもたらす世界金融不安定化の問題に対する有効な解決策は、法的規制ではなく、信認のある中央銀行が発行するステーブルコインを流し込み、それをブロックチェーン決済の基軸通貨にすること}であると考えられる。前節で検討したように、CBSは既存のいかなる民間発行ステーブルコインよりも信認度が高く、投機的ではないブロックチェーン決済の大部分を置き換えることが期待できる。ブロックチェーン決済の大部分が中央銀行管理下の通貨に置き換われば、民間発行ステーブルコインの影響力が弱まり世界金融安定化に資することが期待できる。

また、中央銀行管理下の通貨をブロックチェーンの基軸通貨にするという目標は、FRBやIMFをはじめとする金融当局の意図と整合的である。金融当局は金融当局管理下にない暗号通貨によるブロックチェーン決済が拡大することを懸念しており、たとえばFacebookによるLibra~\cite{libra}が規制されたことや、エルサルバドルのビットコイン法定通貨化に対してIMFが見直し勧告を出したことは記憶に新しい~\cite{imfelsalvador}。現在のところ金融当局は法的規制によって問題を解決しようとしているが、非中央集権的な暗号通貨を法で規制することができない以上、積極的に中央銀行管理下の通貨をブロックチェーンの基軸通貨にするほうがより有効な解決策であると考えられる。

ここでインターネットという非中央集権的システムがいかにして人々にとって有益なものになったかを振り返りたい。インターネットの基盤技術は非中央集権的で違法コンテンツを匿名ネットワークやダークウェブに流通させることは可能であるが、大多数の人々が日常的にアクセスするインターネット空間が違法コンテンツで埋め尽くされているわけではない。なぜかというと、大多数の人々はGoogleやMicrosoftなどの検索エンジン、TwitterやFacebookなどのSNSを通じてインターネット空間にアクセスしており、それらのプラットフォームは健全なビジネスを展開するために、違法コンテンツを弾き人々に有益な情報を提供するインセンティブを持っているからである。インターネットという非中央集権的システムが有益な場所になったのは、違法な行動を法的に規制したからではなく(非中央集権的システムを法で規制することには限界がある)、良いプラットフォームが成長し大多数の人々がそれらの良いプラットフォームを通じてインターネットにアクセスするようになったためである~\footnote{同様の現象はブロックチェーンでもすでに起きつつある。たとえばNFTのマーケットプレースを提供するプラットフォームであるOpenSeaは、不適切なNFTや盗まれたNFTなど規約違反のNFTを検索結果から排除している。多くのユーザはブロックチェーン上で直接NFTを売買するわけではなく、OpenSeaなどのプラットフォームを介して売買を行っており、プラットフォームは健全なビジネスを展開するために不適切なNFTを排除するインセンティブを持っている。その結果、不適切なNFTがブロックチェーン上に存在することは規制しようがなくても、それらのNFTは代表的なプラットフォームから排除されることで価値を失っていく。非中央集権的であるブロックチェーン自体は規制しようがなくても、その上に良いプラットフォームが成長し大多数のユーザがそれらのプラットフォームを通じてアクセスするようになることで有益な場所になっていく。
  
ところで、特定の企業や国家によって運営されるプラットフォームは中央集権的システムであり、ブロックチェーンがプラットフォームによって支配される世界になるとブロックチェーンのそもそもの価値が失われるのではないかという議論がある。著者の意見では、ブロックチェーンが完全に非中央集権的システムとして繁栄するという想定は幻想であるように思われる。なぜかと言えば、ブロックチェーンが繁栄するためにはその上でビジネスが成功しなければならず、ビジネスが成功するのであれば中央集権的なプラットフォームが出現し市場の需要を集約して支配していくからである。\textbf{ブロックチェーンの将来を構想するとき、非中央集権的なDAOのみが繁栄するなか法的規制によって不適切なコンテンツを排除していく姿を想定するよりは、中央集権的なプラットフォームが繁栄してそれらのプラットフォームがビジネスインセンティブによって不適切なコンテンツを排除していく姿を想定するほうが現実的であるように思われ、当局の規制強化および産業育成もその方向性で行われるべきである。}なお、結果的に中央集権的なプラットフォームが支配する世界になるからといってブロックチェーンの非中央集権性が意味を失うわけではない。インターネットがまさにそうであったように、基盤技術が非中央集権的であることによって参入がオープンになり中央集権的システムでは起きえない技術革新が起こりうる。}。同様に考えると、ブロックチェーンという非中央集権的システムを有益なものにするために、中央銀行管理下の通貨を積極的にブロックチェーンに流し込みそれを基軸通貨にするアイディアには一定の合理性があると考えられる。日本は日本円という信認度の高い通貨を持っており、これを実行できる数少ない国のひとつである。


\subsection{基軸通貨発行特権}

CBSがブロックチェーン決済の基軸通貨になった場合、日本円はブロックチェーン決済における基軸通貨的役割を持つことになる。本節では基軸通貨の基本的な性質とその発行国が持つ基軸通貨発行特権について説明する。

一般に基軸通貨~\cite{monetarytheory,robert2004international}とは国際間の決済に広く用いられる通貨のことをいい、現在の基軸通貨は米ドルである。アメリカによって発行された米ドルはアメリカ国外で広く使用されており、たとえば日本とフィリピンが貿易する際の決済に米ドルが使用されたり、エクアドルやエルサルバドルのようなドル化された国々~\cite{calvo2002dollarization,selgin2005currency}で法定通貨として使用されている。

基軸通貨発行国であるアメリカは基軸通貨発行特権を持ち、膨大な貨幣発行益を手に入れている。日本が海外から必要商品を輸入したい場合には、代金支払いに必要な外貨を入手する必要があり、輸出または外貨借り入れによって外貨を調達する必要があるが、基軸通貨発行国であるアメリカは米ドルを刷るだけでよい。相手国が米ドルを受領するかぎりアメリカは米ドルを刷って輸入代金支払いに当てることができる。実際アメリカは恒常的な貿易赤字を抱えている。

この基軸通貨発行特権は非基軸通貨国に対して不公平を生み出す。たとえばドル化を採用しているエルサルバドルが100万米ドルを国内に流通させている状況を考える。この100万米ドルはどこから来たものかといえば、元をたどればエルサルバドルが100万米ドル分の輸出を(直接的または間接的に)アメリカに対して行って手に入れたものであるが、その100万米ドルはアメリカが刷ったものである。つまり100万米ドルの貨幣発行益はエルサルバドルではなくアメリカが手に入れている~\footnote{この不公平な状況を解消するには、エルサルバドルのような途上国が自国通貨主権を回復し、自国通貨の貨幣発行益を手に入れられるようにすることが必要である。ところが途上国は経済規模が小さく不安定なため、ドル化やドルペッグに頼ることなく自国通貨の通貨価値を安定させることが難しい。この問題を解決するために、著者はスマートコントラクトを利用したAlgorithmic Central Bankによって途上国が自国通貨主権を回復する手法を提案している~\cite{johnlawcoinacb}。}。

このように見るとアメリカが基軸通貨発行特権によって不当な利益を得ているように思われるが、アメリカは国際的な決済に広く用いられるのに十分なだけの米ドルを供給する責任を同時に負っていることも忘れてはならない。仮にアメリカが貿易赤字を解消するために輸入を過度に緊縮した場合、国際的な米ドルの流動性不足が発生して経済が混乱することになる。この点を指摘したのがトリフィンのジレンマである~\cite{bordo2019triffin}。トリフィンのジレンマの主張は、基軸通貨発行国は国際的な基軸通貨の流動性需要を満たすように基軸通貨を供給しなければならないが、そのとき基軸通貨発行国は貿易赤字を抱えることになり、貿易赤字は基軸通貨としての信認低下をもたらすというジレンマを抱えているというものである。ただし、著者の見解ではこれはジレンマではない。アメリカは米ドルの国際的な流動性需要を満たすように米ドルを供給しなければならず、実際に恒常的な貿易赤字を抱えているが、これは基軸通貨国として正常な状況であり、貿易赤字であるという事実のみによって基軸通貨としての信認低下が起きることはないと考えられる。基軸通貨国の貿易赤字(適切な流動性供給が行われているかぎり通貨の信認低下は起きない)と非基軸通貨国の貿易赤字(通貨の信認低下が起きうる)は本質的に別物であることに留意する必要がある。またマクロに見れば、基軸通貨国が貿易赤字であるということは、それ以外の非基軸通貨国は総合的に貿易黒字で外貨を蓄積しているということであり正常な状態と言える~\footnote{基軸通貨国が貿易黒字になることのほうが問題であると言える。}。

まとめると、基軸通貨発行国は国際的な決済に広く用いられるのに十分なだけの通貨を供給する責任を負っており、そのプロセスを通じて膨大な貨幣発行益を手に入れることができる。現在の基軸通貨は米ドルであり、通貨の使用には慣性の法則が働くことを考えれば~\cite{goldberg2010international}、いったん成立した基軸通貨を別の通貨が置き換えるのは並大抵のことではない。一方で\textbf{ブロックチェーン決済の分野は、Bitcoin、Ethereum、民間発行ステーブルコインなどさまざまな非法定通貨が群雄割拠している状態であり、基軸通貨の地位を獲得した法定通貨はまだ存在しない}。CBSがブロックチェーン決済の基軸通貨になった場合、日銀はその十分な供給の責任を負うと同時に貨幣発行益を手に入れることができ、日本円の国際競争力を高めることができる。

\section{主要な論点}

\subsection{法定通貨担保型ステーブルコインとの違い}

CBSは、イメージとしては日銀が法定通貨担保型ステーブルコインを発行するようなものであるが、USDTやJPYCのような民間発行の法定通貨担保型ステーブルコインとCBSとの間には決定的な違いが存在する。民間発行の法定通貨担保型ステーブルコインは兌換保証のために100\%の担保を抱えておく必要があり、したがって貨幣発行益はまったく発生せず~\footnote{兌換保証が必要なため担保を運用することも許されない。前述のようにUSDTを発行するTether社は流通するUSDTに対して100\%担保を保有していないのではないかという疑惑がかけられている。}、手数料収入によってビジネスを成立させる必要がある。また担保を保有する企業が倒産する可能性がありカウンタパーティリスクが存在する。これに対して\textbf{CBSは担保を保有する必要がない}。なぜなら日銀は円を発行する権限を持っているからである。$X$円が預金されて$X$CBSコインを発行する時点で、預金された$X$円は何かに使ってもよいし、円供給が増加することによるインフレや円安が心配であれば消せばよい。$X$CBSコインの兌換が要求された時点では、日銀はつねに$X$円を刷って提供することができる。$X$円を刷ったり消したりする操作はコンピュータ上の数字を増減させるだけの操作であり、日銀のバランスシートを拡大・縮小するだけの操作である~\cite{mmt2016wray}。ある時点でCBSが1000兆CBSコイン流通している状況では、日銀は1000兆円の貨幣発行益を手に入れている。これは米ドルが国際的な決済手段として100兆米ドル流通している状況においてアメリカは基軸通貨国として100兆米ドルの貨幣発行益を手に入れていることに相当する。

貨幣発行権を持たない民間主体が発行する法定通貨担保型ステーブルコインと貨幣発行権を持つ主体が発行する法定通貨担保型ステーブルコインは本質的に異なる。

\subsection{ブロックチェーン決済の基軸通貨になることはできるか}

CBSはパブリックブロックチェーン上のERC20トークンとして発行されるため、世界中のユーザが非中央集権的にトランザクションを作成できる。米ドルなどを通じた海外送金と異なり、銀行口座を持つ必要はなく決済は少額の手数料で即時に完了する。この点は既存の暗号通貨と同様である。

ブロックチェーン決済の基軸通貨となるためには既存の暗号通貨に対して明らかな優位性が必要である。第一に、非投機的な決済用途では貨幣価値が安定していることが重要であり、ステーブルコインはBitcoin、Ethereumなどの非ステーブルコインに対して優位性を持つ。第二に、上で述べたように、民間発行の法定通貨担保型ステーブルコインと比較すると、CBSは円の発行権限を持つ中央銀行によって発行されるコインでありカウンタパーティリスクが圧倒的に小さい。CBSは実質的に日本円と同等であり、暗号通貨担保型ステーブルコインやアルゴリズム型ステーブルコインよりも信認度が高い。したがって非投機的なブロックチェーン決済用途において\textbf{CBSは既存の暗号通貨に対して明らかな優位性}があり基軸通貨の地位を得られる可能性はあると思われる。

\subsection{中央銀行デジタル通貨(CBDC)との違い}

Bank of Englandをはじめ各国の中央銀行が構想している中央銀行デジタル通貨(CBDC)~\cite{cbdc1,cbdc2}とCBSは異なる概念である。

\begin{description}
\item[パブリックブロックチェーンかプライベートブロックチェーンか] 技術的観点でいえば、\textbf{CBDCはプライベートブロックチェーンへの円展開であるのに対して\footnote{技術的にはプライベートブロックチェーンは古典的な分散データベースと等価であり、CBDCを実現するのにブロックチェーンは必要ない。実際、2021年に行われた調査によると~\cite{ubssurvey}、71\%の中央銀行がCBDCは通常のデータベースで実現できるためブロックチェーンは必要ないと回答している。}、CBSはパブリックブロックチェーンへの円展開である}。
\item[解決される社会問題は何か] \textbf{CBDCが解決するのは「日本国民の決済の電子化」であるのに対して、CBSが解決するのは「パブリックブロックチェーンへの円勢力拡大(貨幣発行益の獲得)」「真のステーブルコイン供給によるブロックチェーン経済圏の発展」「民間暗号通貨の影響力を抑制することによる金融安定化」である}。
\item[国境を越えた利用を想定するか] CBDCは自国内決済システムを想定しているが、CBSは国境を越えたブロックチェーン決済での利用を想定している。
\item[KYC~\cite{parra2017kyc}を要求するか] CBDCはAML(Anti-Money Laundering)/CFT(Combating the Financing of Terrorism)~\cite{verdugo2011compliance}の対策としてトランザクションを追跡する必要があるためKYCを要求する。これに対してCBSはKYCを要求しない。第一の理由は、ある国の中央銀行が国境を越えて別の国の国民や企業に対して口座作成を許可したりKYCの情報を収集することは法的およびプライバシー的に非常に困難であるためである。仮に法とプライバシーの問題をクリアできたとしても、国境を越えてKYCを適切に管理するプラットフォームを構築・運用するのは非常に大きなコストがかかる。各国のCBDCやエルサルバドルが提供するBitcoinウォレットであるChivoウォレットがKYCを要求できるのは利用が自国内の国民や企業に制限されているためである。第二の理由は、\textbf{そもそもCBSの目標はKYCが要求されない非中央集権的なブロックチェーン決済の分野に中央銀行管理の通貨を流し込むこと}だからである。好むと好まざるとにかかわらず市場には非中央集権的な決済への強い需要が存在し、上で述べたようにそれを法的に規制するには限界があるため、中央銀行管理の通貨を流し込むことが世界金融安定化への有効な対策になると考えられる。KYCを要求した場合、それは既存の銀行決済やCBDCの延長にすぎず本稿が提示する問題の解決にならない。
\item[貨幣発行益が生み出されるか] CBDCは紙幣の単なる「電子マネー化」にすぎず、紙幣の管理を中央銀行の台帳データベースに変えたものである。すでに円を利用している日本国民の決済手段を電子化するだけなので、円の通用範囲が広がるわけではなく貨幣発行益は生み出されない。これに対してCBSはブロックチェーン決済の基軸通貨となることで、まだ円を利用していない世界中のユーザに円の通用範囲を拡大するので貨幣発行益を生み出す。
\end{description}

このようにCBSとCBDCは異なるものであり両立可能なものである。

\subsection{AML/CFTのリスク}

CBSはKYCを要求しないため\textbf{最大のリスクはAML/CFT対策}であると考えられる。AML/CFTに対しては技術面でも法制面でもさまざまな対応策が検討されているが、2019年に金融庁主導のもとで実施された「ブロックチェーンを用いた金融取引のプライバシー保護と追跡可能性に関する調査研究」~\cite{blockchainprivacy}に示されているように、ミキシング、ステルスアドレス、リング署名、ゼロ知識証明~\cite{sasson2014zerocash}などトランザクション匿名化の技術が多数開発されており、根本的にはパブリックブロックチェーン上でのトランザクション追跡は技術的に不可能と考えるべきである。したがって、日銀がCBSを発行することはマネーロンダリングやテロ資金供与の温床となりうる通貨を日本が提供することを意味し、国際的な批判にさらされるリスクがある。

AML/CFTに関して以下の点を指摘しておきたい。

\begin{itemize}
\item CBSがマネーロンダリングやテロ資金供与の温床になることは問題であるがこれは暗号通貨一般の問題であって、CBSがあってもなくてもマネーロンダリングやテロ資金供与は発生する。仮にCBSを行わなかった場合、マネーロンダリングやテロ資金供与が別の暗号通貨で起きるだけの話である。AML/CFTができないからという理由でCBSを行わないという立場を取ったところで、(中央銀行の責任回避にはなっても)エコシステム全体にとっては問題の解決になっていない。AML/CFTが完全にできないからという理由で世界金融安定化の目標を捨てるのは失うもののほうが大きいように思われる。
\item 日本を含む多くの先進国が十分な担保を保有することを条件としてすでに民間発行ステーブルコインを容認している。つまり、\textbf{AML/CFTができないにもかかわらず、法定通貨をパブリックブロックチェーン上にトークン化することを先進国の中央銀行はすでに容認している}。AML/CFTの実行性の可否の観点から言えばCBSと民間発行ステーブルコインに違いはない。
\item パブリックブロックチェーンで非中央集権的に実行されるトランザクションに対してAML/CFTを行うことは不可能であるが、国家の監督下にあるウォレットや取引所で実行されるトランザクションに対してKYCとAML/CFTを行うことは可能である。それらのウォレットや取引所を利用するインセンティブを作ることによって、多くのトランザクションに対してAML/CFTを行うことが可能になる。たとえばエルサルバドルはビットコインを法定通貨化しているが、多くの国民はビットコインのパブリックブロックチェーンに直接トランザクションを作っているわけではなく、エルサルバドル政府が管理するChivoウォレットを利用しておりChivoウォレットではKYCとAML/CFTが行われている。
\item AML/CFTは程度の問題であって、CBSを発行せずとも中央銀行が発行する紙幣はマネーロンダリングやテロ資金供与に利用されうる。
\item 将来的に暗号通貨が広く利用されるようになった世界において、AML/CFTはトランザクション以外のレイヤで対策されるべき問題に変質する可能性がある。現時点でブロックチェーンにおけるAML/CFTが問題になるのは、「貨幣のトランザクションは追跡できるものである」という前提が人々の間に存在し、よって「マネーロンダリングやテロ資金供与のトランザクションは禁止できる」と人々が考えているからであるが、暗号通貨の世界では匿名的なトランザクションを作成可能であり、この前提はすでに破綻している。追跡できないトランザクションが当然のように存在する世界になればこの前提が変化する可能性がある。たとえばインターネットによってテロ組織が情報交換することは可能だが、インターネットの存在が問題視されることが少ないのは、それがもたらす便益の大きさと、インターネットが当然のものとなり追跡できない通信が存在することを人々が認めているからである。AML/CFTは、古典的な管理された銀行間取引に対して非中央集権的な暗号通貨取引が出現した際の過渡期的現象とも捉えられる。ただし、仮にAML/CFTが過渡期的現象であるとしても現在はその過渡期にあるのだから現在AML/CFTが問題視される事実に変わりはない。
\end{itemize}

\subsection{貨幣発行益を手に入れることは望ましいことか}

CBSによって日本円がブロックチェーン決済の基軸通貨になれば、日銀は十分な供給を提供する責任を負うと同時に貨幣発行益を手に入れることができるが、基軸通貨発行によって貨幣発行益を得ることが望ましいことかどうかについて考察しておきたい。

重要な事実は、CBSを発行して1000兆円が流通した場合、それがなければ発生しえなかった1000兆円分の価値交換が生じているので、(マネーロンダリングやテロ資金供与などの望ましくない価値交換がないとすれば)大きな価値が生み出されているという点である。問題なのは、基軸通貨発行によって貨幣発行益を手に入れることではなく、その貨幣発行益を基軸通貨発行国がどう再分配するかである。日本がCBSの貨幣発行益を国富として蓄えれば不当な利益を得ていると受け止められる可能性があるが、1000兆円の貨幣発行益を途上国支援に使うと約束すれば国際的な理解が得やすいものと思われる。仮に貨幣発行益の1000兆円全部を途上国支援に使ったとしても、日本の国際的なプレゼンスと日本円の競争力は高まることが予想され日本の国富につながることが期待される。このように\textbf{CBSの貨幣発行益の是非は望ましい再分配の問題}として議論されるべきである。

\subsection{CBSはどのブロックチェーンで発行されるべきか}

CBSを発行するパブリックブロックチェーンとしては、現在最も規模の大きなEthereumチェーンが有効であると考えられるが、どれか1つに絞る必要はなく、Polygon、Astar、Solanaなど複数のブロックチェーンに発行してもよい。実際、JPYCは複数のブロックチェーンに対して日本円担保型ステーブルコインを発行している。仮に将来的にブロックチェーンの勢力図が変化し、主流がEthereumチェーンからSolanaチェーンに変わった場合には、ユーザはEthereumチェーン上のCBSをいったん円に交換し、その円をSolanaチェーン上のCBSに変換すればCBSを移行することができる。日銀がCBSと円との兌換を保証しているため、チェーンの勢力が弱まったとしても1CBSコインはつねに1円の価値を持っており、ユーザがCBSを失う不利益を被ることはない。たとえるなら、聖徳太子の一万円札を銀行に持っていけば最新の一万円札に交換してもらえることに相当する。

視点を変えると、国家がCBSを発行することは、従来の紙幣と硬貨という形態に加えてブロックチェーンという形態でも貨幣を発行することに相当する。つまり\textbf{CBSは発行する貨幣形態の現代的な自然な延長}にすぎない。将来的にブロックチェーンが社会の基盤プラットフォームになれば、いずれの国家もCBSを発行しない状況のほうが考えにくいと思われる。

\subsection{他国の反応}

日本がCBSを発行した場合、他国の反応としては、他国が日本のCBSを規制してくる可能性と、他国が追随してCBSを発行する可能性の2通りが想定できる。

CBSは基軸通貨発行特権をもたらすものであり、マネーロンダリングやテロ資金供与に利用される可能性もあるため、短期的には他国あるいはIMFが日本に対して停止を勧告する可能性がある。ところが、本稿で繰り返し述べてきたように、信認のあるCBSをブロックチェーンに流し込むことこそが暗号通貨が世界金融に及ぼす悪影響を低減させブロックチェーン経済圏を拡大させるのに有効であること、かつ、それを実行した国は基軸通貨発行特権が得られることを考えれば、\textbf{長期的にいずれの国もCBSを発行しないという状態は考えにくい}。よって長期的には他国が追随してCBSを発行すると考えられるが、以下の2点を指摘しておきたい。

\begin{itemize}
\item 日本がCBSを発行しない場合は他国が先に発行する可能性があり、通貨の使用には慣性の法則が働くことを考えれば先行した国ほど基軸通貨としての地位を獲得する可能性が高い。よって\textbf{10年後にいずれかの先進国がCBSを発行するかどうかは日本が国家戦略として検討するに値する問題であり、仮にその答えがYesである場合、日本はCBSを発行するべきである}。10年後にいずれかの先進国がCBSを発行するかどうかを検討するには、CBSが存在する世界と存在しない世界を比較するのが有効である。「CBSがブロックチェーン決済の基軸通貨になることで、暗号通貨決済の大部分が中央銀行・IMFの管理下にある通貨で実行されている世界」と「暗号通貨決済の大部分が中央銀行・IMFの管理下になくかつ規制不可能な民間暗号通貨で実行されている世界」を比較すれば、世界金融安定化の観点から見て確実に前者のほうが望ましい。
\item 他国がCBSを発行した場合(たとえば米ドルやユーロがパブリックブロックチェーンにトークン化された場合)、それらが日本のCBSより優位になり日本のCBSの流通が限定的になる可能性がある。この場合日本は基軸通貨特権は得られないことになるが、\textbf{人類の通貨システムは確実に進化しており日本が先駆的にそれに貢献したことになる}。通貨システムを進化させるという観点では、むしろ複数のCBSが競争することでより良いステーブルコインが生まれることが望ましいと言える~\cite{hayek2009denationalisation}。
\end{itemize}

\section{結論}

本稿ではCBSを提案し、日銀がパブリックブロックチェーン上で日本円をトークン化してステーブルコインを発行した場合、どのような可能性があるかについて考察した。非中央集権的なブロックチェーンには法的規制が効果的に機能しない以上、ブロックチェーン経済の決済通貨を健全に発展させるためには、信認のあるCBSをブロックチェーンに流し込むことが有効であると考えられ、いずれかの国がCBSを発行するのは時間の問題であるように思われる。貨幣形態の観点から言えば、CBSは発行する貨幣形態の現代的な自然な延長(紙幣と硬貨からブロックチェーンへ)にすぎない。10年後にいずれかの先進国がCBSを発行するかどうかは日本が国家戦略として検討するに値する問題であり、先行者利益の存在を考慮すれば、仮にその答えがYesである場合、日本はCBSを発行するべきである。

CBSの論点やリスクを挙げればきりがないが、重要なことは、経済産業省の資料にあるように~\cite{meti}、\textbf{「不利なビジネス環境を押しつけられるルールテイカーの立場に陥らないよう自ら戦略を持って行動」するためには他国に追従する・既存の枠組内で可能なことをやるだけでは不十分で、10年後に起きうる世界の姿を想定してそれをドライブする立場に立つという視座で議論することである}。

\bibliography{whitepaper.bib}
\bibliographystyle{unsrt}

\end{document}
