\documentclass[dvipdfmx,a4paper]{article}
\usepackage{amsmath,amssymb,multicol,indentfirst,a4wide,tablefootnote}
\usepackage[dvipdfmx]{graphicx}

\def\tab{\hskip 2em}
\renewcommand\arraystretch{1.4}
\sloppy

\title{\textbf{Central Bank Stablecoin}}
\author{Kentaro Hara\footnote{Email: xharaken@gmail.com. Opinions are my own and not the views of my employer.}}
\date{2022 November}

\begin{document}

\maketitle

\begin{abstract}

This paper proposes \textbf{\textit{Central Bank Stablecoin (CBS)}} and discusses the benefits and the risks. The idea of CBS is that the central banks of advanced countries tokenize their fiat currencies (e.g., USD, JPY) as stablecoins on public blockchains. The paper shows that CBS has the following benefits:

\begin{itemize}
\item CBS will become a truly trustworthy stablecoin on public blockchains because it is issued by the central bank of an advanced country. CBS will enable blockchain transactions that are not yet happening with existing stablecoins issued by private sectors and thus expand the blockchain economy.
\item The authorities are worried that stablecoins issued by private sectors are repeating crashes and increasing instability of the financial market. If CBS becomes a key currency of the blockchain economy, the impact of stablecoins issued by private sectors will diminish and it will help stabilize the financial market.
\item The central bank that issues CBS can obtain seigniorage. Also their fiat currency can increase international competitiveness by becoming a key currency of the blockchain economy.
\end{itemize}

As explained in the paper, CBS is different from Central Bank Digital Currency (CBDC).

\end{abstract}

\section{Proposal}

To simplify the explanation, the paper assumes that FRB (the central bank of the United States) issues a stablecoin backed by USD. The central bank of other countries can use the same technology to issue a stablecoin backed by their fiat currency.

The idea of \textbf{\textit{Central Bank Stablecoin (CBS)}} is very simple. The central bank deploys a smart contract on a public blockchain and issues a stablecoin as an ERC20 token~\cite{erc20token}. Specifically, the central bank exchanges $X$ CBS coins with $X$ USD. When a user deposits $X$ USD to the central bank, the central bank issues $X$ CBS coins to the user. When a user redeems $X$ CBS coins to the central bank, the central bank returns $X$ USD to the user. 1 CBS coin is equivalent to 1 USD because CBS is just a tokenized version of USD.

\section{Benefits}

\subsection{Provide a truly trustworthy stablecoin to the blockchain economy}

In theory, the main functions of money are distinguished as: a medium of exchange, a unit of account, and a store of value~\cite{davies2010history,ferguson2008ascent}. To meet these functions as money, price stability is a mandatory requirement. For example, Bitcoin works as a medium of exchange but does not work as a unit of account and a store of value because the volatility is high. In El Salvador, where Bitcoin is used as legal tender, retail stores are using USD to display the price of their goods (i.e., Bitcoin is not working as a unit of account) and accepting a payment in Bitcoin using the BTC / USD exchange rate at that point (Bitcoin is working as a medium of exchange). The government of El Salvador is providing a service that converts the exchanged Bitcoin to USD immediately with no fee to protect users from the volatility risk (i.e., Bitcoin is not working as a store of value). High volatile cryptocurrencies work as a target of speculations but do not work as a payment method of non-speculative, real-world transactions.

It is important to note that \textbf{a stabilized currency is crucial to grow a non-speculative, real economy}. For example, developing countries tend to adopt dollarization~\cite{calvo2002dollarization,selgin2005currency} or dollar pegging to stabilize their domestic currency because they need a stabilized currency as a payment method for non-speculative, real economic transactions. Currently, speculative transactions are dominating the blockchain business (e.g., NFTs, DeFi). A truly trustworthy stablecoin will play a key role to expand the blockchain economy to non-speculative transactions and thus grow the economy to the next level.

Stablecoins issued by private sectors can be classified~\cite{arner2020stablecoins,moin2020sok} into fiat-collateralized stablecoins (e.g., USDT~\cite{tether}, JPYC), crypto-collateralized stablecoins (e.g., MakerDAO~\cite{makerdao}, Curve~\cite{curve}) and algorithmic stablecoins (e.g., Terra~\cite{terra}, Empty Set Dollar~\cite{emptysetdollar}, Frax~\cite{frax}). However, it is hard to say that these stablecoins are trustworthy enough.

\begin{itemize}
\item Fiat-collateralized stablecoins hold 100\% collateral in some fiat currencies. Fiat-collateralized stablecoins need gatekeepers that hold the collateral and the gatekeepers may be regulated or go bankrupt. For example, people are questioning whether Tether really holds adequate USD reserves to back the circulating USDT, but Tether has been failing at providing a promised audit about it. Recently stablecoins are getting more and more scrutinized and regulated in advanced countries and the counterparty risk about the gatekeepers is not a hypothetical concern.
\item Crypto-collateralized stablecoins use cryptocurrencies as the collateral. The challenge is that cryptocurrencies have high volatility and thus crypto-collateralized stablecoins need over-collateralization. For example, MakerDAO asks users to keep a 150+\% collateralization ratio (e.g., when a user reserves ETH that is worth 150 USD, the user can issue up to 100 coins). If the collateralization ratio drops to less than 150\%, the reserved ETH is forced to default and liquidated. This actually happened when the price of ETH dropped from \$1300 to \$84 in 2018 and many users lost their reserved ETH. Also crypto-collateralized stablecoins are normally operated by DAO. It means that the monetary protocols are subject to change by opinions of users who have the majority of the governance tokens. This community-based governance makes the future shape of the protocols unpredictable.
\item Algorithmic stablecoins attempt to stabilize the currency price by controlling the total coin supply without holding any collateral. In reality, however, many algorithmic stablecoins have failed at stabilizing the currency price and crashed (e.g., Empty Set Dollar, Terra)~\footnote{The author is proposing JohnLawCoin~\cite{johnlawcoin}, yet another attempt to stabilize the currency price of an algorithmic stablecoin.}.
\end{itemize}

CBS is a truly trustworthy stablecoin because it is issued by the central bank and the central bank guarantees the conversion between 1 CBS coin and 1 USD. CBS is expected to be more trusted than any stablecoins issued by private sectors and help grow non-speculative transactions in the blockchain economy.

\subsection{Reduce the impact of stablecoins issued by private sectors}

Empty Set Dollar crashed in 2020 December, and Terra crashed in 2022 May. The crash of Terra had a broad impact not only on the market of algorithmic stablecoins but also the market of stablecoins in general. The co-founder of Terra was prosecuted for violating the capital market rules. FRB's Chairman Jerome Powell said in a speech that the central bank should play a leading role in regulating the issuers of stablecoins and floated wider crypto regulation that could impact digital asset wallets. The trend for these regulations is understandable because the market believes that stablecoins are stable but in reality, stablecoins issued by private sectors have risks, as described in the previous section. Assuming that the market of stablecoins grows more in the future, the authorities are worried that the crash of the stablecoins might disrupt the world-wide financial market at a non-negligible scale, just like the Great Recession in 2008 produced bad loans and caused a financial crisis.

It is important to remember that \textbf{legal regulations will not solve the problem}. It is possible to enforce legal regulations to fiat-collateralized stablecoins because they have gatekeepers that hold collateral. However, it is impossible to enforce legal regulations to crypto-collateralized stablecoins and algorithmic stablecoins because they are fully decentralized and have no gatekeepers to regulate. The increased demand for stablecoins will expand the market of these not-regulatable stablecoins in the future.

\textbf{A more effective solution to this problem is to let the central banks of advanced countries issue stablecoins and make the stablecoins a key currency of the blockchain economy}. CBS is more trustworthy than any stablecoins issued by private sectors and has a good chance of becoming a standard currency for non-speculative transactions in the blockchain economy. This will weaken the impact of stablecoins issued by private sectors and help reduce the risks in the financial market.

CBS aims at becoming a key currency of the blockchain economy. This goal is aligned with the intention of the authorities including FRB and IMF. The authorities are worried about the situation where more and more transactions happen with currencies that are not under control of the central banks. For example, Facebook's Libra~\cite{libra} was regulated by the authorities of multiple countries. IMF published the recommendation~\cite{imfelsalvador} that El Salvador should stop using Bitcoin as legal tender. These authorities are trying to solve the problem by legal regurations but it will not work because it is impossible to regulate decentralized cryptocurrencies. CBS provides a more effective solution by making the central bank's stablecoins a key currency of the blockchain economy.

\subsection{An analogy}

The internet has succeeded in building a valuable ecosystem while being fully decentralized. Due to the decentralized nature of the underlying technologies, it is technically possible to upload illegal content to anonymous networks and the dark web. However, the internet is not yet broken; i.e., the internet is a valuable place for the majority of people. This is because the majority of people are accessing the internet through search engines like Google and Microsoft and social media like Facebook and Twitter, and these platforms are incentivized to kick out illegal content and provide valuable content to run a healthy business. The key observation is that \textbf{the internet has become a valuable place not because the authorities have regulated illegal content (it is impossible to fully regulate content on a decentralized system) but because the majority of people have accessed the internet through good platforms}. A similar movement should happen in the blockchain economy~\footnote{The movement is already emerging in the blockchain economy. For example, OpenSea, a platform that provides a marketplace to exchange NFTs, filters out NFTs that do not meet their terms of service (e.g., illegal NFTs, stolen NFTs) from their search result. OpenSea is incentivized to do this to run a healthy business, and the majority of users sell and buy NFTs using OpenSea. Illegal NFTs will lose value because they will be kicked out from popular platforms. The majority of users access content through good platforms, and the good platforms enable the blockchain economy to become a valuable place while being decentralized. This is similar to how the internet has evolved.}. It is not effective to regulate stablecoins issued by private sectors. A more effective solution is to let the central banks create truly trustworthy stablecoins and let the majority of people use the trustworthy stablecoins. CBS achieves this.

One possible counter argument is that platforms owned by particular organizations (including CBS owned by particular central banks) are centralized systems and that ruins the value of decentralized blockchains. The author's opinion is that the view that the blockchain economy thrives as a fully decentralized ecosystem is a myth. The reason is as follows. For the blockchain economy to thrive, there must be business opportunities. If there are business opportunities, centralized platforms emerge and aggregate the demand of the market. \textbf{The centralized platforms dominate the business of the blockchain economy and the platforms filter out illegal content}. The idea that CBS dominates non-speculative transactions of the blockchain economy and reduces the impact of stablecoins issued by private sectors is aligned with the model.

Even if centralized platforms dominate the business, it does not ruin the value of the decentralized blockchains. The decentralized nature of blockchains enables open innovations that would not happen with centralized systems, just like the internet has demonstrated.


\subsection{Monetary sovereignty and seigniorage}

CBS aims at becoming a key currency of the blockchain economy. This section explains the basic properties of a key currency and the monetary sovereignty of the central bank that issues CBS.

Generally speaking, a key currency~\cite{robert2004international} refers to a currency broadly used for international transactions. The current key currency is USD. USD is used broadly in international transactions outside the United States. For example, Japan and the Philippines use USD as a payment currency for their trade. El Salvador and Ecuador adopt dollarization~\cite{calvo2002dollarization,selgin2005currency} and use USD as legal tender.

The important observation is that the United States has monetary sovereignty and obtains a massive amount of seigniorage. If Japan wants to import some goods from another country with USD, Japan needs to obtain the USD by exporting some goods or borrowing. On the other hand, the United States can obtain the USD just by minting because the United States has the monetary sovereignty. The United States can import goods from another country ``for free'' as long as USD is accepted by the country. In fact, the United States has been facing a trade deficit for many years. This is unfair. For example, imagine the situation where El Salvador circulates 1 million USD in their country. The key fact is that El Salvador obtained the 1 million USD by exporting goods and services (directly or indirectly) to the United States, and the United States minted the 1 million USD ``for free''. In other words, the seigniorage goes to the United States, not El Salvador~\footnote{To resolve the unfair situation, El Salvador needs to have monetary sovereignty so that they can issue their own currency and obtain seigniorage. However, it is challenging for developing countries like El Salvador to stabilize their own currency due to the limited economy size and political instability. To solve this problem, the author proposes a way for developing countries to obtain monetary sovereignty using an Algorithmic Central Bank~\cite{johnlawcoinacb} powered by a smart contract.}.

It may look like the United States is abusing the monetary sovereignty and obtaining the seigniorage unfairly. However, remember that the United States is responsible for providing adequate USD enough to be used for world-wide international transactions. If the United States overly reduces import to eliminate the trade deficit, it will cause a shortage of USD's liquidity in the international economy. According to the Triffin dilemma~\cite{bordo2019triffin}, the key-currency country is responsible for issuing the key currency enough to meet the demand for the international economy, but then the country will surface a significant level of trade deficit and thus the key currency will lose trust. The author's opinion is, however, that this is not a dilemma. The United States must provide adequate USD to meet the demand for the international economy and thus is indeed surfacing the trade deficit, but this is a natural situation and the mere fact that the United States is surfacing the trade deficit does not indicate that USD loses trust as a key currency. It is important to note that a trade deficit of the key-currency country is not a problem as long as the country keeps providing an appropriate amount of liquidity. It is qualitatively different from a trade deficit of a non-key-currency country, which is a problem. From the macroeconomic perspective, the fact that the key-currency country surfaces a trade deficit indicates that other non-key-currency countries surface a trade surplus and save the key currency, which is a healthy situation. It is more problematic that the key-currency country surfaces a trade surplus.

In summary, the key-currency country is responsible for providing an adequate amount of liquidity to be used in international transactions and can obtain seigniorage. Today's key currency is USD, and considering that inertia works in the key currency~\cite{goldberg2010international}, it is not realistic to replace the key currency with another currency. On the other hand, \textbf{there is no fiat currency that gets the position of the key currency in the blockchain economy yet}. BTC, ETH and stablecoins issued by private sectors are competing with each other. A country that succeeds in getting the position of the key currency of the blockchain economy with CBS can obtain a significant amount of seigniorage and improve the international competitiveness of their fiat currency.

\section{Discussion}

\subsection{Differences from fiat-collateralized stablecoins}

CBS is similar to fiat-collateralized stablecoins in the sense that the central bank guarantees a conversion between 1 CBS coin and 1 USD. However, there is a fundamental difference between CBS and fiat-collateralized stablecoins. Fiat-collateralized stablecoins issued by private sectors need to hold 100\% collateral to guarantee the conversion of circulating coins. Thus they produce no seigniorage~\footnote{Private sectors that issue fiat-collateralized stablecoins are not encouraged or allowed to invest the collateral but in reality, some private sectors are doing it. For example, people are questioning whether Tether really holds adequate USD reserves to back the circulating USDT but Tether has not yet provided a promised audit.} and their business relies on commission fee. Also there is a counterparty risk since the private sectors may go bankrupt. On the other hand, \textbf{CBS does not need to hold collateral} because the central bank has the monetary sovereignty to issue USD. When a user deposits $X$ USD to the central bank, the central bank issues $X$ CBS coins to the user. The central bank may use the $X$ USD for some purposes, or if the central bank is concerned about inflation or currency depreciation, the central bank may burn the $X$ USD. When a user returns $X$ CBS coins to the central bank, it is guaranteed that the central bank can issue $X$ USD. No collateral is needed because the central bank has the monetary sovereignty. The operations to issue or burn $X$ USD are just an operation to increase or decrease numbers in the central bank's computer~\cite{mmt2016wray}.

Imagine a situation where 1 trillion CBS coins are circulating in the market. At this point the central bank has obtained seigniorage that is worth 1 trillion USD. This corresponds to the situation where when 1 trillion USD is circulating in the international economy, the United States has obtained seigniorage that is worth 1 trillion USD as a key-currency country.

Fiat-collateralized stablecoins issued by private sectors that do not have the monetary sovereignty are fundamentally different from CBS issued by the central banks that have the monetary sovereignty.

\subsection{Can CBS become a key currency of the blockchain economy?}

CBS is issued as an ERC20 token on a public blockchain. Any users can create transactions in a fully decentralized manner. Users do not need to hold a bank account to use CBS and it will improve financial inclusion. Transactions complete immediately and the transaction fee is small.

For CBS to become the key currency of the blockchain economy, there must be a clear benefit compared to existing cryptocurrencies. First of all, non-speculative transactions require price stability, and stablecoins are preferred over non-stablecoins like BTC and ETH. Second, as described in the above, CBS is issued by the central bank that has the monetary sovereignty and thus the counterparty risk is lower compared to fiat-collateralized stablecoins issued by private sectors. CBS will be more trusted than any existing stablecoins issued by private sectors, including crypto-collateralized stablecoins and algorithmic stablecoins. In this way, \textbf{CBS has clear benefits compared to existing cryptocurrencies} and has potential to get the position of the key currency of the blockchain economy.

\subsection{Differences between CBS and CBDC}

The Bank of England and other central banks have been exploring Central Bank Digital Currency (CBDC)~\cite{cbdc1,cbdc2}. CBDC and CBS are different concepts as follows:

\begin{description}
\item[Does it use a public blockchain or a private blockchain?] From the technical perspective, \textbf{CBDC issues the fiat currency on a private blockchain~\footnote{Technically a private blockchain is equivalent to a traditional distributed database system. CBDC does not need blockchains. According to the survey in 2021~\cite{ubssurvey}, 71\% of the central banks responded that CBDC can be implemented with a traditional distributed database and does not need blockchains.} whereas CBS issues the fiat currency on a public blockchain}.
\item[What are the goals?] The goal of CBDC is to digitize payment transactions inside the country. On the other hand, \textbf{the goals of CBS are to 1) expand the currency area of the fiat currency to public blockchains and obtain seigniorage, 2) grow the blockchain economy by providing a truly trustworthy stablecoin, and 3) stabilize the financial market by reducing the impact of stablecoins issued by private sectors}.
\item[Is it used outside the country?] CBDC is a payment system inside the country. On the other hand, CBS is deployed on a public blockchain and used by any users in the world.
\item[Does it require KYC (Know Your Customer)~\cite{parra2017kyc}?] CBDC requires KYC because it needs to track transactions for AML (Anti-Money Laundering) and CFT (Combating the Financing of Terrorism)~\cite{verdugo2011compliance}. On the other hand, CBS does not require KYC for the following reasons. First, from the legal and privacy perspective, it will not be allowed for the central bank to collect KYC information of users living in other countries. Even if the legal and privacy concerns are addressed, it will be really expensive to build and maintain a platform that works across multiple countries. Remember that CBDC can require KYC because the usage of the wallet is restricted to inside the country (e.g., Chivo wallet of El Salvador). Second, \textbf{the goal of CBS is to enable the central banks to have more control on the fully decentralized blockchain economy where KYC is not required}. There is a strong demand for fully decentralized transactions, and the authorities are worried that the expansion of the blockchain economy may disrupt the financial market. As described in the above, an effective solution to this problem is to let the central banks of advanced countries issue stablecoins and make (one of) the stablecoins work as the key currency of the blockchain economy rather than tightening regulations because it is inherently not possible to regulate decentralized stablecoins. CBS is a solution for the blockchain economy where KYC is not required.
\item[Does it produce seigniorage?] CBDC is just an electronic version of physical bank notes and only digitizes payment transactions inside the country. CBDC does not aim at expanding the currency area and  thus does not produce seigniorage. On the other hand, CBS issues coins on a public blockchain. CBS expands the currency area to world-wide blockchain users and produces seigniorage.
\end{description}

CBDC and CBS are different things and they can coexist.

\subsection{AML and CFT}

\textbf{The biggest challenge of CBS will be AML and CFT because CBS cannot require KYC}. A lot of research has been conducted to track blockchain transactions and enforce AML and CFT. However, there are many technologies to anonymize blockchain transactions such as mixing, stealth address, ring signature, zero knowledge proof~\cite{sasson2014zerocash}, and it is technically impossible to track transactions on public blockchains and enforce AML and CFT~\cite{zhang2019security}. It means that the central bank that issues CBS has a risk of encountering international criticism because it provides a currency that enables money laundering and terrorist financing.

The following points should be highlighted about AML and CFT:

\begin{itemize}
\item AML and CFT are not a problem of CBS but a problem of decentralized cryptocurrencies in general. Money laundering and terrorist financing happen regardless of CBS. If the central bank does not issue CBS, money laundering and terrorist financing will just end up happening with other cryptocurrencies. Not issuing CBS will not help reduce money laundering and terrorist financing in the ecosystem. Given that CBS helps stabilize the financial market and benefits the ecosystem, CBS will be an overall win for the ecosystem.
\item Multiple advanced countries already allow private sectors to issue fiat-collateralized stablecoins. In other words, \textbf{the central banks of advanced countries already allow deploying their fiat currencies onto public blockchains via private sectors even though AML and CFT cannot be enforced}. From the perspective of AML and CFT, CBS is no different from stablecoins issued by private sectors.
\item It is not possible to enforce AML and CFT for decentralized transactions that directly happen on public blockchains but it is possible to enforce AML and CFT for transactions that happen through wallets and intermediaries (e.g., currency exchangers) that can be controlled by the government. One option is that the government can create incentives for users to use government-provided wallets and intermediaries and thus make it possible to enforce AML and CFT on most transactions. For example, El Salvador adopts Bitcoin as legal tender. Most citizens in El Salvador are using the Chivo wallet provided by the government instead of creating transactions directly on Bitcoin's public blockchain. The government can require KYC for the Chivo wallet's users and enforce AML and CFT.
\item AML and CFT are a matter of degree. For example, physical bank notes issued by the central bank can also be used for money laundering and terrorist financing.
\item Currently the authorities are trying to enforce AML and CFT with the assumption that monetary transactions can be traced and thus money laundering and terrorist financing can be banned. However, cryptocurrencies enable users to create anonymous transactions and the assumption is already broken. In a future world where cryptocurrencies are more commoditized, the authorities will realize that monetary transactions are not traceable and AML and CFT cannot be enforced at the transaction layer. AML and CFT can be viewed as an ``intermediate'' problem that emerges only in the transition period from the traditional transactions between banks (which are traceable) to the anonymous transactions of cryptocurrencies (which are not traceable). A similar phenomenon was observed on the internet. The internet can be used for terrorists to exchange information but the authorities of democratic countries  do not try to regulate the internet for that reason. This is because the authorities recognize that privacy of communication should be guaranteed and communication over the internet is not traceable. Similarly, it is possible that the authorities may update their view about AML and CFT over time as cryptocurrencies are more commoditized.
\end{itemize}

\subsection{Seigniorage}

The central bank that issues CBS is responsible for providing adequate coin supply and can obtain seigniorage. This section discusses whether it is the right thing for the central bank to obtain seigniorage.

The important fact is that in the situation where 1 trillion CBS coins (which are worth 1 trillion USD) are circulated, the central bank has succeeded in creating new transactions that would not have happened otherwise. Ignoring undesirable transactions like money laundering and terrorist financing, this is a good thing. \textbf{The problem is not the fact that the central bank obtains seigniorage but how the central bank redistributes the seigniorage}. If the central bank saves the seigniorage as national wealth, it may be perceived that the central bank obtains profits unfairly. On the other hand, if the central bank uses the seigniorage to support developing countries, it will be able to get international understanding more easily. Even if the central bank uses the whole seigniorage for developing countries, it will have a win because the country can improve the presence in the international relations and the competitiveness of their fiat currency.

\subsection{What blockchains should CBS use?}

There are many public blockchains in the world. Ethereum is the most popular, widely used public blockchain and will be a first, good target for deploying CBS. However, CBS can be deployed on multiple blockchains, just like USDC is deployed on Ethereum, Solana, Avalanche, TRON etc. If the popular blockchains change in the future, users can just migrate their CBS coins from one blockchain to another blockchain. For example, if the most popular blockchain changes from Ethereum to Solana, users can convert CBS coins on Ethereum with USD and then convert the USD with CBS coins on Solana. Since the central bank guarantees the conversion between CBS coins and USD, 1 CBS coin has the value of 1 USD always regardless of the popularity of the blockchain on which the 1 CBS coin lives. Users do not need to worry about losing the 1 CBS coin when the blockchain loses popularity. An analogy is that when you bring an old bank note to a bank, the bank will change the old bank note with a new bank note.

The central bank already issues their fiat currency as a form of physical bank notes and metallic coins. CBS is (just) a proposal to issue the fiat currency as a form of tokens on public blockchains in addition to the tranditional forms. \textbf{CBS can be viewed as a natural and modern extension of the monetary format}. Assuming that blockchains will become a foundational platform in society in the future, it is hard to imagine that none of the governments issues CBS on public blockchains.

\subsection{Reactions from other countries}

Imagine that the central bank of one advanced country issues CBS. There are two possible reactions from other countries. The first possible reaction is that CBS of the country is regulated by other countries. The second possible reaction is that other countries issue CBS as well and compete.

In the short term, the first reaction is likely to happen. CBS brings a significant amount of seigniorage. Also there is a good reason to regulate CBS because CBS may be used for money laundering and terrorist financing. Other countries or the authorities such as IMF may make a recommendation to pause CBS. In the long term, however, the second reaction will happen because \textbf{it is hard to imagine that none of the advanced countries issues CBS in the long term} for the following reasons; 1) CBS will provide a truly trustworthy stablecoin and help expand the blockchain economy, 2) CBS will stabilize the financial market by crowding out stablecoins issued by private sectors and 3) the central bank that issues CBS first can obtain seigniorage. In the long term, multiple advanced countries may issue CBS and compete. It is a good thing because the competition will improve the currency system of CBS~\cite{hayek2009denationalisation}.

Considering that inertia works in the key currency~\cite{goldberg2010international}, it is hard to replace an already established key currency with another currency. It means that the first country that issues CBS has more chances to succeed. This creates an incentive for all advanced countries to explore CBS. \textbf{It is strategically important for each country to assess if any advanced country will issue CBS in the next 10 years, and if the answer is yes, the country has an incentive to issue CBS first to get the seigniorage}.

If CBS happens in the next 10 years, CBS will become the key currency of the blockchain economy and the central banks and IMF will be able to get most of the non-speculative blockchain transactions under their control. If CBS does not happen in the next 10 years, most blockchain transactions will be out of control of the central banks and IMF. The former is more desirable and thus the author's opinion is that CBS will happen in the next 10 years.

\section{Conclusion}

This paper proposed CBS and discussed the benefits and the risks. The idea is that the central banks of advanced countries tokenize their fiat currencies (e.g., USD, JPY) as stablecoins on public blockchains.

Regulations do not work effectively for fully decentralized systems like blockchains. A more effective solution to grow the blockchain economy is to let the central banks issue stablecoins and make the stablecoins the key currency of the blockchain economy. The stablecoins will be more trusted and used than any existing stablecoins issued by private sectors and help the central banks get more control over the blockchain economy. This reduces the risk of financial crisis caused by untrustworthy stablecoins issued by private sectors. From the monetary format perspective, CBS is just a natural and modern extension from the physical bank notes and metallic coins to tokens on smart contracts.

For all the central banks of advanced countries, it is strategically important to investigate whether CBS will happen in the next 10 years or not. It is possible to create an infinite list of questions and concerns about CBS but the most important point is to envision what \textit{should} happen in the next 10 years and push the world forward with the vision. Innovation requires challenges. The central bank that issues a successful CBS first can obtain seigniorage, which is a reward for the pioneer.

\bibliography{whitepaper.bib}
\bibliographystyle{unsrt}

\end{document}
