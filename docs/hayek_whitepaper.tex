\documentclass[dvipdfmx,a4paper]{article}
\usepackage{amsmath,amssymb,multicol,indentfirst}
\usepackage[dvipdfmx]{graphicx}

\def\tab{\hskip 2em}
\renewcommand\arraystretch{1.4}
\sloppy

\title{\textbf{Experimenting with Hayek's Denationalisation of Money in Developing Countries}}
\author{Kentaro Hara\footnote{Email: xharaken@gmail.com, GitHub: https://github.com/xharaken/john-law-coin. Opinions are my own and not the views of my employer.}}
\date{2025 January}

\begin{document}

\maketitle

\begin{abstract}

Friedrich Hayek published \textit{``Denationalisation of Money: The Argument Refined''} in 1978. Hayek argued that governments have always been inclined to print money and cause inflation. Hayek proposed that national governments should drop the monopoly of issuing money, allow private businesses to issue private currencies, and let them compete. Since the success of the private currencies depends on how broadly they are accepted, the private businesses are incentivized to issue a stable currency. The country will obtain a more stable currency as a result of the currency competition.

\textbf{Hayek's proposal is highly debatable in multiple points but can be valid in developing countries that have been actually failing to issue stable national currencies or struggling with getting out of dollarization.} For those countries, there is a good reason to believe that currency competition will produce currencies that are more stable than the ones issued by unstable governments. This paper analyzes the economic history of Zimbabwe and Venezuela, which have caused hyperinflation multiple times and failed to issue stable national currencies, and Ecuador and El Salvador, which still adopt dollarization. With dollarization, all the seigniorage is ``stolen'' by the United States.
  
This paper proposes experimenting with the currency competition in these developing countries with the goal of letting them obtain a stable national currency. The concrete proposal is simple:

\begin{enumerate}
\item \textbf{All the government needs to do is to create a law to allow private currencies and implement a few incentives and rules to encourage competition.}
\item Once the right incentives and rules are implemented, the ecosystem will implement private currencies and drive competition. A private currency with the greatest stability in value will become dominant.
\item Once a stable private currency is established, we move the monetary sovereignty from the private business to the government (e.g., the government may purchase the business). The government gets a stable national currency.
\end{enumerate}

\end{abstract}

\section{Introduction}

\subsection{Hayek's proposal}

Friedrich Hayek published \textit{``The Denationalisation of Money''} in 1976~\cite{hayekbook2} and then a revised version \textit{``Denationalisation of Money: The Argument Refined''} in 1978~\cite{hayekbook}. Hayek argued that governments have been always inclined to print money and cause inflation: \textit{``Once governments are given the power to benefit particular groups or sections of the population, the mechanism of majority government forces them to use it to gain the support of a sufficient number of them to command a majority. The constant temptation to meet local or sectional dissatisfaction by manipulating the quantity of money so that more can be spent on services for those clamouring for assistance will often be irresistible. Such expenditure is not an appropriate remedy but necessarily upsets the proper functioning of the market.''}~\cite{hayekbook} Hayek proposed that \textbf{national governments should drop the monopoly of issuing money, allow private businesses to issue their own forms of money, and let them compete}. Competitions in an open market will favor currencies with the greatest stability in value and produce more stable currencies than the ones monopolized by national governments.

In this paper, \textit{national currency} refers to a currency issued by national governments. \textit{Private currency} refers to a currency issued by private businesses.

\subsection{Counter arguments}

Hayek's argument is highly debatable. Several economists criticized the arguments. David H. Howard argued that the assumption that the market will choose the most stable currency is questionable, and that Hayek's regime of competitive currencies may just result in an establishment of a new monopoly similar to the national currency~\cite{howard1977denationalization}. Milton Friedman et al also questioned the assumption that if the current legal obstacles to the production of private currencies were removed, stable private currencies would become dominant~\cite{friedman1986has}. Most importantly, no countries have ever attempted Hayek's proposal so far. There is no empirical evidence to believe that currency competition will result in a more stable currency.

Another debatable point is that Hayek was assuming that any inflation is bad and that a good money should realize zero inflation. If Hayek's proposal were implemented, people would prefer holding zero-inflating currencies over mildly-inflating currencies. This makes the zero-inflating currencies become dominant. However, modern macroeconomics generally believes that mild inflation is desirable to stimulate and grow the economy. The Phillips curve~\cite{stock2008phillips} states that inflation and unemployment have a trade-off relationship; higher inflation is associated with lower unemployment, and vice versa. For this reason, inflation targeting, which has been adopted by various countries successfully~\cite{svensson2010inflation}, sets the inflation expectation to a mildly positive value (e.g., 2\%). Also, Milton Friedman proposes a $k$ percent rule, where the central bank should increase the money supply by a constant percentage rate every year, irrespective of business cycles. Hayek's assumption that zero-inflating currencies are desirable is questionable.

To summarize, it is not desirable or realistic to adopt Hayek's proposal in developed countries that have established stable national currencies (e.g., United States, Europe, Japan) for multiple reasons:

\begin{itemize}
\item It is highly skeptical that Hayek's proposal will result in a more stable currency.
\item Modern macroeconomics generally believes that mild inflation is more desirable than zero inflation.
\item There are no incentives for governments to lose control over their national currencies.
\end{itemize}

\subsection{Proposal}

However, \textbf{the situation is very different in developing countries that have been failing to provide stable national currencies}. In some countries, governments do not have the ability to issue stable national currencies due to political instability. For example, Zimbabwe~\cite{imf2009zimbabwe,kramarenko2010zimbabwe,hanke2009measurement,nkomazana2014overview,coomer2011hyperinflation} and Venezuela~\cite{reilly2020hyperinflation,pittaluga2021political,mondal2023venezuela,miller2019hyperinflation,huertas2019hyperinflation,manzano2014venezuela,kulesza2017inflation} experienced hyperinflation multiple times in the past decades. For these countries, there is a good reason to believe that currency competition among private businesses will result in a more stable currency than the ones issued by unstable governments.

Another potential target is developing countries whose governments are matured but adopt dollarization~\cite{quispe2006official,edwards2001dollarization}, such as El Salvador~\cite{alonso2023salvador,sohn2019economic,castillo2011macroeconomic,alvarez2023cryptocurrencies} and Ecuador~\cite{beckerman2002crisis,beckerman2001dollarization}. Dollarization is an effective strategy to circulate a stable currency in the country but has the problem that all the seigniorage is ``stolen'' by the United States. Currency competition among private businesses may be useful to establish a stable currency and let the country get out of dollarization.

This paper proposes experimenting with Hayek's Denationalisation of Money in those developing countries. \textbf{The end goal is to enable the governments to issue a stable national currency and get out of dollarization.}

\textbf{The concrete proposal is simple: All the government needs to do is to create a law to allow private currencies and implement a few incentives and rules to encourage competition.} Once the right incentives and rules are implemented, the ecosystem will implement private currencies and drive competition.

To clarify, the currency competition is not the end state. As described above, it is highly skeptical that the currency competition will result in better currencies than the ones issued by trusted governments. The currency competition is an intermediate step to obtain a stable private currency and/or get out of dollarization. The end state should be that governments are trusted and issue stable national currencies (just like developed countries). In other words, the currency system should evolve in the following order:

\begin{enumerate}
\item A government is failing to issue a stable national currency and/or struggling with getting out of dollarization.
\item The government leverages the currency competition in the ecosystem and obtains a stable private currency.
\item We move the monetary sovereignty of the stable private currency from the private business to the government (e.g., the government purchases the business). The government issues a stable national currency.
\end{enumerate}

This paper describes how to move step 1 to 2 and step 2 to 3. Developed countries that have already reached step 3 are outside the scope of the paper.

\subsection{Paper organization}

This paper is organized as follows. Section 2 summarizes Hayek's \textit{``Denationalisation of Money: The Argument Refined''} and clarifies a few differences between Hayek's original proposal and this paper's proposal. Section 3 overviews the economy of Zimbabwe, Venezuela, Ecuador and El Salvador as the potential target of the experiment. Section 4 describes the concrete proposal and the possible outcome in the ecosystem. Section 5 concludes.

\section{Hayek's Denationalisation of Money}

\subsection{Summary of Hayek's points}

This section summarizes the key points of Hayek's \textit{``Denationalisation of Money: The Argument Refined''}~\cite{hayekbook}.

\subsubsection*{Point 1: National currencies without the gold standard are full of history of inflation because governments are likely to abuse the power of issuing money}

At the very beginning of the book, Hayek says: \textit{``The main result at this stage is that the chief blemish of the market order which has been the cause of well-justified reproaches, its susceptibility to recurrent periods of depression and unemployment, is a consequence of the age-old government monopoly of the issue of money. I have now no doubt whatever that private enterprise, if it had not been prevented by government, could and would long ago have provided the public with a choice of currencies, and those that prevailed in the competition would have been essentially stable in value and would have prevented both excessive stimulation of investment and the consequent periods of contraction.''}

Hayek investigates the history of various national currencies without the gold standard and argues that it is largely a history of inflation engineered by governments. Governments have the exclusive power to issue and regulate money, making people accept any amount of money as they wish. Governments are likely to abuse the power because they are incentivized to benefit particular groups or sections of the population to gain their support to command the majority in the election. There is a constant temptation for governments to meet their financial needs by issuing money, resulting in inflation.

\subsubsection*{Point 2: Any inflation is bad}

Hayek believes that all inflations, including mild ones, are bad. Hayek says: \textit{``All inflation is so very dangerous precisely because many people, including many economists, regard a mild inflation as harmless and even beneficial.''} Hayek believes zero inflation is desirable.

\subsubsection*{Point 3: Currency competition favors a stable currency and realizes zero inflation}

Currency competition works with the following mechanism:

\begin{enumerate}
\item People prefer holding appreciating currencies rather than depreciating currencies.
\item If the currencies are appreciating, private businesses issuing the currencies are incentivized to issue more money.
\item Thus a stable (i.e., not appreciating or depreciating) currency becomes dominant as a result of the currency competition.\footnote{It is worth noting that Gresham's law does not apply here. Gresham's law states that “bad money drives out good money". For example, if there are two forms of commodity money in circulation, which are accepted by law as having the same face value, the more valuable commodity will gradually disappear from circulation. People prefer holding the more valuable commodity and spending the less valuable commodity. This behavior happens because the two commodities have different real values but they are accepted by the same face value. In other words, it happens only when a fixed exchange rate is enforced by law. On the other hand, in the world of Hayek's currency competition, nothing enforces a fixed exchange rate between currencies, and the exchange rate fluctuates depending on the real values of the currencies. Thus the behavior that drives Gresham's law does not happen.}
\end{enumerate}

Hayek is not saying that currency competition will eliminate inflation and deflation completely. If the fear of an impending world catastrophe (e.g., war) happens, nothing could prevent a general fall in the prices. Hayek is saying that currency competition will eliminate inflation and deflation caused by monetary factors: \textit{``Money is the one thing competition would not make cheap, because its attractiveness rests on it preserving its `dearness'.''}

\subsubsection*{Point 4: Private businesses are incentivized to issue a stable currency}

The success of private currencies depends on how broadly they are accepted. Given that currency competition favors a stable currency, private businesses are incentivized to issue a stable currency.

Hayek states that currency competition is a more efficient way to stabilize currencies than the gold standard: \textit{``The threat of the speedy loss of their whole business if they failed to meet expectations would provide a much stronger safeguard than any that could be devised against a government monopoly. Competition would certainly prove a more effective constraint, forcing the issuing institutions to keep the value of their currency constant (in terms of a stated collection of commodities), than would any obligation to redeem the currency in those commodities (or in gold). And it would be an infinitely cheaper method than the accumulation and the storing of valuable materials.''} \textit{``Convertibility is a safeguard necessary to impose upon a monopolist, but unnecessary with competing suppliers who cannot maintain themselves in the business unless they provide money at least as advantageous to the user as anybody else.''}

\subsubsection*{Point 5: The concept of national currencies is not inevitable or desirable}

Hayek envisioned the world where private currencies circulate beyond the border of one country. People and merchants use whatever currencies that make the most sense to them in each circumstance. It means that multiple currencies (which have different denominations and units) are mixed in one country. That might be inconvenient but the benefit of having stable currencies will outweigh the cost. The currency area~\cite{frankel1998endogenity} will be determined independently from the country border. Some currencies may be used only in a local area in one country. Other currencies may be used across multiple countries. Hayek says: \textit{``There is indeed little reason why, apart from the effects of monopolies made possible by national protection, territories that happen to be under the same government should form distinct national economic areas which would benefit by having a common currency distinct from that of other areas.''} This is close to the world cryptocurrencies like Bitcoin have realized.

\subsubsection*{Point 6: Banks should work with 100\% reserves (i.e., narrow banking)}

The currency competition assumes that private businesses have full control over the amount of private currencies they issue. The assumption breaks if banks start credit creation. It is too complicated for the private businesses to supervise credit creation done by banks, implement ``monetary policies'' to tune the interest rate, and work as the lender of last resort when the banks go bankrupt. Note that the private businesses and the banks may be in different countries. To make the currency competition work, credit creation needs to be banned, meaning that banks should work with 100\% reserves (i.e., narrow banking~\cite{pennacchi2012narrow,kay2009narrow}).

\subsubsection*{Point 7: Monetary policies are no longer needed}

If banks work with 100\% reserves, monetary policies are no longer needed. The concept of interest rate control is gone.

Hayek assumes that monetary policies are harmful: \textit{``What we should have learned is that monetary policy is much more likely to be a cause than a cure of depressions, because it is much easier, by giving in to the clamour for cheap money, to cause those misdirections of production that make a later reaction inevitable, than to assist the economy in extricating itself from the consequences of overdeveloping in particular directions. The past instability of the market economy is the consequence of the exclusion of the most important regulator of the market mechanism, money, from itself being regulated by the market process.''} \textit{``With the central banks and the monopoly of the issue of money would, of course, disappear also the possibility of deliberately determining the rate of interest. The disappearance of what is called 'interest policy' is wholly desirable.''}

\subsection{Discussion}

Hayek's vision is interesting but highly debatable. Realistically speaking, no governments have attempted Hayek's proposal so far. Governments do not have incentives to adopt Hayek's vision for multiple reasons:

\begin{itemize}
\item Regarding Point 2, modern macroeconomics generally believe that a mild inflation is desirable to stimulate and grow the economy, and many developed countries use inflation targeting.
\item Regarding Point 3, as Milton Friedman~\cite{friedman1986has} and David H. Howard~\cite{howard1977denationalization} pointed out, it is skeptical that currency competition will produce private currencies that are more stable than national currencies.
\item Regarding Point 5, governments have no incentive to abandon their national currencies (unless they can get benefits that outweigh the cost, like Euro and dollarization). At the very least, governments have no incentives to let their economy use whatever currencies issued by private businesses inside or outside their country.
\item Regarding Point 6 and 7, it is highly debatable if narrow banking is a better system. For example, in 2019, FRB denied approval for narrow banking in the United States, claiming that they would interrupt FRB's implementation of monetary policies and thereby potentially have harmful effects on the general financial stability~\cite{grasselli2019broad}.
\end{itemize}

From the realistic point of view, the author does not agree with Hayek's stated goals. Alternatively, the author thinks that Hayek's idea of currency competition has great potential when it is used in a different way; i.e., the author proposes to use the currency competition to help untrusted governments currently failing to issue a stable national currency get a stable private currency. Then the governments can purchase the private business and get a stable national currency.

\textbf{The author fully respects Hayek's idea of currency competition but it is important to clarify that the goals are opposite. Hayek's goal was to \textit{denationalize} money. On the other hand, the author's goal is to help untrusted governments obtain \textit{nationalized} money by leveraging the currency competition of denationalized money as an intermediate step.}

\section{Case studies}

The potential targets of this paper's proposal are governments that are failing to issue a stable national currency and/or adopting dollarization. This section studies Zimbabwe, Venezuela, Ecuador and El Salvador.

\subsection{Zimbabwe}

Zimbabwe acquired its independence in 1980. Zimbabwe introduced the Zimbabwean dollar as an official national currency, replacing the Rhodesian dollar in equivalence. In 2003, the land reform programme confiscated thousands of commercial farms from white farmers, and the farms were distributed to the landless black majority. Encouraged by the government promises and the apparent lack of police intervention, hundreds of white-owned farms were invaded and taken over by war veterans~\cite{nkomazana2014overview,coomer2011hyperinflation}. This caused the downfall of agriculture, which was the backbone of the economy. Production dropped and workers were laid off. Zimbabwe was producing less but spending more. According to the IMF report~\cite{imf2009zimbabwe}, Zimbabwe's economy experienced a 14\% drop in GDP in 2008, on top of a 40\% cumulative decline during 2000–2007. An unemployment rate reached 80\%. An annual inflation rate is estimated to have peaked at about 500 billion percent in September 2008~\cite{hanke2009measurement}. This led Zimbabwe to be the first hyperinflationary economy of the 21st century.

During the hyperinflation period, the government denominated the Zimbabwean dollar three times (ZWD $\to$ ZWN $\to$ ZWR $\to$ ZWL). In 2006, ZWD was denominated by $1/10^3$ to ZWN. In 2008, ZWN was denominated by $1/10^{10}$ to ZWR. In 2009, ZWR was denominated by $1/10^{12}$ to ZWL. Nevertheless, the government continued printing money. For example, in 2006, Zimbabwe's central bank announced that it would print larger bills to buy foreign currencies. The central bank printed 21 trillion Zimbabwe dollars to pay off debts owed to IMF. In 2009, the use of foreign currencies was legalized, and the government effectively abandoned the Zimbabwean dollar~\cite{kramarenko2010zimbabwe}. Since then, Zimbabwe had used a combination of foreign currencies, mostly US dollars and South African Rand~\cite{coomer2011hyperinflation}.

In 2019, Zimbabwe introduced a new Zimbabwean dollar (called RTGS) and announced that all foreign currencies were no longer legal tender. However, this caused hyperinflation again. In 2020, IMF estimated that an annual inflation rate reached 837\%~\cite{imf2022zimbabwe}. In 2024, Zimbabwe announced a new currency called Zimbabwe Gold, which is backed by hard assets including foreign currencies, gold and other precious metals.

\subsection{Venezuela}

Countries in Latin America frequently have experienced episodes of hyperinflation due to political instability and populist regimes~\cite{pittaluga2021political}. In Venezuela, Hugo Chávez came to power in 1998 and started the Bolivarian Revolution. Chávez expanded the state ownership of the means of production and moved away from market-based production~\cite{miller2019hyperinflation}. Venezuela was one of the highest oil producing countries in the South African region, and the economy was heavily relying on oil exports. The oil exports accounted for 80\% of the total exports in 1995 - 2002~\cite{manzano2014venezuela}. Chávez took an expansive fiscal policy and spent considerably more on social programs intended to address poverty and inequality. These ranged from subsidies for those on low incomes to health services. The Venezuelan government was able to afford the high cost of these programs only as long as the oil price stays high~\cite{reilly2020hyperinflation}.

The expansionary fiscal policies of Chávez, combined with a substantial external debt, increased inflationary pressure in 2008. With declining oil prices in 2014, the economic problems were exacerbated. The fiscal deficit reached 23\% of GDP in 2017, and the government printed money to meet their financial needs~\cite{mondal2023venezuela}. This created a cycle for hyperinflation~\cite{kulesza2017inflation}. In 2018 alone, GDP shrank by close to one-fourth. The fiscal deficit reached 15\% of GDP. The monetary base grew by 73,000+ percent, driven by the government's spending needs. 9 out of 10 Venezuelans are living in poverty. Child malnutrition reached 15\% in some states~\cite{huertas2019hyperinflation}. IMF estimated the annual inflation rate to have reached 65000\% in 2018.

Bolivar is Venezuela's official currency. Due to the repeated inflations, the government denominated Bolivar three times (VEB $\to$ VEF $\to$ VES $\to$ VED). In 2008, VEB was denominated by $1/10^3$ to VEF. Bolivar was pegged to US dollars but did not stay stable despite attempts to institute capital controls. The central bank tried to stick to the pegged exchange rate. However, since Bolivar was actually overpriced, people used parallel exchange rates in a black market despite a ban. In 2018, VEF was denominated by $1/10^5$ to VES. In 2021, VES was denominated by $1/10^6$ to VED. IMF estimated the annual inflation rate in 2023 was 337\%. Venezuela's economy has undergone extensive currency substitution, and the majority of transactions happen in US dollars.

\subsection{Ecuador}

In the 1990s, Ecuador started reforms but suffered several large external shocks and natural disasters, and then culminated the decade with the disruptive economic and financial crisis. The fiscal structure has traditionally been dependent on the revenues of oil and a few other commodities. Large fiscal deficits and increasing external debt led to imbalances that became unsustainable with the decline of the world oil prices and El Niño's devastating impact in 1998. The consumption-based poverty incidence rose from 34\% in 1995 to 46\% in 1998. In 1999, Ecuador's national currency Sucre lost 67\% of its foreign exchange value. Their GDP dropped by 7.3\%. A fiscal deficit reached 5\% of the GDP. A large part of the bank deposits was frozen. The government caused a default on external debt payments~\cite{beckerman2002crisis,quispe2006official,beckerman2001dollarization}.

These accelerated a flight from the national currency and de-facto dollarization. In January 2000, Ecuador officially decided to adopt US dollars as legal tender to restore the price stability. The full dollarization succeeded. The GDP started to recover and marked a 5.1\% growth in 2001. The annual inflation rate dropped to 7.9\% in 2003 and 2.7\% in 2004, converging to the US dollar's inflation rate~\cite{quispe2006official}. As of 2024, Ecuador still adopts full dollarization.

\subsection{El Salvador}

After reaching a peace agreement in the early 1990s resolving a civil war, El Salvador started a reform to rebuild and stabilize the economy. The reform included simplification of the tax structure, reprivatization of the financial system, and financial and trade liberalization. In 1993, the central bank adopted a fixed exchange rate regime for their national currency Colon with respect to US dollars (8.75 Colones are equal to one US dollar) to minimize the exchange rate risks and promote price stability. Thanks to the reform, the GDP growth averaged 6\% between 1990 and 1995, and 3.7\% between 1998 and 2000~\cite{quispe2006official}. IMF estimated that the annual inflation rate was 4.3\% in 2000.

In 2001, El Salvador adopted full dollarization. Unlike Ecuador, which adopted full dollarization as a response to the economic and banking crisis, El Salvador had enjoyed economic stability and low inflation rates before the dollarization. The motivation was to lower the interest rates further, increase foreign investment, improve financial conditions, and decrease transaction costs in international trade, thereby further accelerating economic growth. The government also pointed out that full dollarization was the logical next step because Colon had been pegged to US dollars since 1993~\cite{quispe2006official}.

More interestingly, in June 2021, El Salvador approved Bitcoin as legal tender in addition to US dollars, becoming the first country in the world that legally accepts cryptocurrencies~\cite{alvarez2023cryptocurrencies,alonso2023salvador}. The motivation for this adoption was not monetary reasons since inflation was controlled by the dollarization and the interest rate was low. 40\% of the resident population of El Salvador was emigrant, of which 88\% resides in the United States, meaning that remittances have a high importance in the country's GDP~\cite{castillo2011macroeconomic}. In 2020, the remittances accounted for 16\% of the GDP. Bitcoin substantially lowered the transaction cost of the remittances. The government launched a Bitcoin wallet called Chivo. To encourage the usage, \$30 in Bitcoin were handed out to all the citizens who downloaded and used the wallet. The commission fee of the transactions done in Chivo was set to zero. 184 Chivo ATMs were built in the country. In 2021, President Bukele announced a plan to create a Bitcoin City with the idea of promoting cryptocurrency mining by harnessing the geothermal energy of volcanoes~\cite{alonso2023salvador}.

\subsection{Analysis}

As discussed above, Zimbabwe and Venezuela have attempted to introduce their own national currencies multiple times but failed due to political instability and weak economy. Due to the repeated hyperinflation and denominations, their national currencies lost trust, and a de-facto dollarization has been on-going in the economy. Ecuador adopted dollarization and got out of the economic and banking crisis. El Salvador adopted dollarization to further stabilize the interest rate and recently started a wild experiment to adopt Bitcoin as legal tender.

Dollarization has costs. With dollarization, the money supply in the country is constrained by the amount of US dollars held by the country. Since the central bank cannot issue any money, there is no seigniorage. The central bank has no discretionary power to enforce monetary policies. The central bank cannot work as the lender of last resort.

The most important cost of dollarization is that monetary sovereignty is on the United States and the seigniorage goes to the United States~\cite{hara2021acb}. When the United States wants to import some goods from another country, they can do so by printing US dollars because they have monetary sovereignty. The United States can import goods from another country ``for free'' as long as US dollars are accepted by the country. In fact, the United States has been facing significant trade deficits for years. On the other hand, when El Salvador wants to import some goods from another country with US dollars, or even when El Salvador simply wants to use US dollars in their country, El Salvador needs to obtain the US dollars by exporting some goods. \textbf{Dollarization is a very unfair system.} For example, imagine the situation where El Salvador circulates 1 million US dollars in their country. The key observation is that El Salvador obtained the 1 million US dollars by exporting goods and services (directly or indirectly) to the United States, and the United States minted the 1 million US dollars ``for free''. The seigniorage goes to the United States, not El Salvador\footnote{The same argument applies for Bitcoin because El Salvador does not have monetary sovereignty to issue Bitcoin.}.

Dollarization is a useful approach to achieve price stability in countries that would otherwise suffer hyperinflation due to political and economic instability. However, dollarization should not be the end state. To achieve economic independence eventually, it is desirable for developing countries to establish their own national currencies and obtain monetary sovereignty.

This paper proposes introducing Hayek's idea of currency competition in these developing countries with the goals of enabling the governments that are failing to issue a stable national currency (e.g., Zimbabwe, Venezuela) to obtain a stable currency and/or the governments to get out of dollarization (e.g., Ecuador, El Salvador) when they wish to do so.

\section{Proposal}

\subsection{Concrete proposal}

\textbf{The concrete proposal is simple: All the government needs to do is to allow private currencies and implement a few incentives and rules to encourage competition}. Specifically, the government needs to do the following three things.

First, the government creates a law and allows private businesses to issue private currencies. At this point, the government does not need to abandon their existing national currency or dollarization. Note that private currencies are private currencies, and not legal tender (yet). Taxes are paid with the national currency or US dollars, not the private currencies. The currency competition can run in parallel with the existing national currency or dollarization.

Second, the government bans credit creation for the private currencies. It is too complicated for the private businesses to supervise credit creation done by banks, implement ``monetary policies'' to tune the interest rate, and work as the lender of last resort when the banks go bankrupt. To avoid the complexity and allow the private businesses to have full control over the amount of the private currencies they issue, credit creation needs to be banned\footnote{Credit creation needs to be banned only for the private currencies. Credit creation can be allowed for national currencies or US dollars because the government can supervise and control the credit creation done by banks.}.

Third, the government implements a few incentives to encourage currency competition. \textbf{Incentives should be implemented in such a way that there is a win-win relationship between the privacy businesses and the government.} Once the right incentives and rules are implemented, the ecosystem will implement privacy currencies and drive the competition. Examples are:

\begin{itemize}
\item Incentives for the private businesses: They may collect a few percent commission fee from each payment transaction. They may collect user data on the payment transactions and use it for marketing purposes. If the private currency is issued with less than 100\% reserves, they may get seigniorage. The government may give them some additional privileges (e.g., corporate tax reduction). For private currencies that are broadly adopted and can be trusted, the government may start accepting taxes with the private currencies.
\item Incentives for the government: The government may ask the private businesses to pay a few percent commission fee for each payment transaction. The government can get a stable currency when the currency competition is successful.
\end{itemize}

It is important to implement these changes at once, not gradually. Hayek says: \textit{``Introduce new currencies at once, not gradually. The other important requirement of government action, if the transition to the new order is to be successful, is that all the required liberties be conceded at once, and no tentative and timid attempt be made to introduce the new order gradually, or to reserve powers of control 'in case anything goes wrong'. The possibility of free competition between a multiplicity of issuing institutions and the complete freedom of all movements of currency and capital across frontiers are equally essential to the success of the scheme. Any hesitant approach by a gradual relaxation of the existing monopoly of issue would be certain to make it fail. People would learn to trust the new money only if they were confident it was completely exempt from any government control. Only because they were under the sharp control of competition could the private banks be trusted to keep their money stable.''}

The currency competition is not the end state. The proposal is to use the currency competition as an intermediate step to help the government obtain a stable private currency and/or get out of dollarization. The next step is to move the monetary sovereignty from the private businesses to the government because the end goal is to enable the government to obtain a national currency\footnote{To get to the end state, the government needs to meet other conditions too; political stability, a reasonable budget deficit, a reasonable GDP growth etc.}. For example, the government can purchase the private businesses that won the currency competition. This will also work as another incentive for the private businesses to participate in the currency competition.

\subsection{Discussion}

\subsubsection{Possible outcome}

No governments in the world have ever attempted currency competition. It is the first attempt in human history and will be sensational enough to attract venture capitals to participate in.

Realistically speaking, the competition will start with private currencies with 100\% reserves because in countries that are failing to issue stable national currencies, it will be challenging to obtain people's trust in other ways. A private business issues one coin in exchange for one US dollar and puts the US dollar in their reserve (where the private business can give a favorite name to the coin). This is no different from dollarization, and the private business cannot obtain seigniorage. The private business makes money by 1) imposing a commission fee on each payment transaction, 2) collecting user data from the payment transactions and using it for their business, and/or 3) purchasing safe assets with the reserves and obtaining investment gains.

As the next step, private currencies that run with partial reserves may appear. The private currencies may run with a managed floating exchange rate regime~\cite{calvo2002fear}, meaning that the private businesses intervene in the market and stabilize the exchange rate of their coins within some range around \$1.0. When the coin price goes far beyond \$1.0, they increase the coin supply by purchasing US dollars and issuing new coins. When the coin price goes far below \$1.0, they contract the coin supply by selling the reserved US dollars and contracting the coins. This currency intervention can be conducted with partial reserves, and 100\% reserves are not always needed. This gives seigniorage to the private businesses.

\subsubsection{Payment infrastructure}

People in the world, including ones in developing countries, are accustomed to QR code payment~\cite{yan2021qr}. Merchants can put the QR code of the private currencies they accept in front of their cash register. Customers can pay using one of the private currencies. Acceptance from merchants and customers defines what private currencies are broadly accepted by the economy. The QR code payment works with mobile applications on the customers (they just need to scan the QR code and hit a pay button), and no special infrastructure is needed on the merchant side. \textbf{Governments do not need to implement heavy infrastructures} like Chivo wallets and ATMs that cover most parts of the country. Governments can leverage the ecosystem to drive the competition.

\subsubsection{Cryptocurrencies}

Private currencies may be issued in the form of cryptocurrencies. The benefit of using public blockchain-based cryptocurrencies is that all the monetary policies are defined in the smart contract and executed transparently. Cryptocurrencies may be more trusted than untrusted governments. However, according to a survey conducted in El Salvador, 46\% of the respondents claimed that they do not understand cryptocurrencies at all~\cite{alonso2023salvador}. As long as private currencies are issued by trusted companies (e.g., well-recognized foreign companies), it is probably sufficient to get people's trust, and cryptocurrencies might not be needed. Either way, the important thing is that various types of private currencies participate in the currency competition, including cryptocurrencies and non-cryptocurrencies issued by trusted companies, and the winner is identified as a result of the competition.

\subsubsection{Worst case scenarios}

Finally, let's analyze what will happen if the currency competition fails. The most likely failure pattern will be that people do not trust and use private currencies. The currency competition fails but there is no harm. The government turns down the competition, and the situation is no worse than the status quo.

The worst case scenario will be that a private business that issues a broadly accepted private currency does some flaud and crashes the value of the private currency. For example, if it turns out that the private currency does not have a 100\% reserve while it was promised to hold a 100\% reserve, it may crash the value. If the private business suddenly increases the money supply for no valid reasons, it may crash the value. If the payment infrastructure is exploited by malicious attackers, it may crash the value. A crash of a broadly accepted private currency has a risk of triggering financial crisis in the economy. It is challenging to prevent these cases completely but the government can reduce the risk by monitoring private businesses issuing broadly accepted private currencies, while the monitoring level should be limited to the extent that does not disturb the currency competition.

\subsection{Relationship with Algorithmic Central Bank}

In 2022, the author proposed Algorithmic Central Bank~\cite{hara2021johnlawcoin,hara2021acb} to enable untrusted governments to issue stable national currencies. The idea was to use an algorithmic stablecoin defined on a public blockchain. Even if the governments are not trusted, smart contracts on a public blockchain execute the monetary policies transparently and can be trusted. Algorithmic Central Bank proposed to implement currency intervention based on a managed floating exchange rate regime with partial reserves so that the exchange rate stays within a configured range, letting the smart contracts run the currency intervention automatically on a public blockchain.

However, Algorithmic Central Bank hit a few challenges: 1) it was challenging to convince governments with this particular solution, 2) the solution requires governments to implement infrastructures like user wallets and ATMs with significant go-to-market efforts, 3) cryptocurrencies bring risks about money laundering and terrorist financing and governments are not willing to take the risks.

After lots of conversations and analysis, the author concluded that the proposal of Algorithmic Central Bank was too prescriptive. \textbf{A better approach is to have the governments allow private currencies, run the currency competition, and let the ecosystem find the right solution through the competition}. This is a lower cost for the governments. The governments do not need to implement heavy infrastructures like user wallets and ATMs. This led to the proposal of this paper.

\section{Conclusion}

Friedrich Hayek published \textit{``Denationalisation of Money: The Argument Refined''} in 1978. Hayek proposed that national governments should drop the monopoly of issuing money, allow private businesses to issue private currencies, and let them compete. Hayek's proposal is highly debatable in multiple points but can be valid in developing countries that have been actually failing to issue stable national currencies or struggling with getting out of dollarization. For those countries, there is a good reason to believe that currency competition will produce currencies that are more stable than the ones issued by unstable governments. This paper analyzed the economic history of Zimbabwe, Venezuela, Ecuador and El Salvador.

This paper proposed experimenting with the currency competition in these developing countries with the goal of letting them obtain a stable national currency. The concrete proposal is simple:

\begin{enumerate}
\item All the government needs to do is to create a law to allow private currencies and implement a few incentives and rules to encourage competition.
\item Once the right incentives and rules are implemented, the ecosystem will implement private currencies and drive competition. A private currency with the greatest stability in value will become dominant.
\item Once a stable private currency is established, we move the monetary sovereignty from the private business to the government (e.g., the government may purchase the business). The government gets a stable national currency.
\end{enumerate}

\bibliography{whitepaper.bib}
\bibliographystyle{unsrt}

\end{document}

