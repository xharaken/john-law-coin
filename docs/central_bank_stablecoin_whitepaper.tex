\documentclass[dvipdfmx,a4paper]{jsarticle}
\usepackage{amsmath,amssymb,multicol,indentfirst,tablefootnote}
\usepackage[dvipdfmx]{graphicx}

\def\tab{\hskip 2em}
\renewcommand\arraystretch{1.4}
\sloppy

\title{\textbf{日銀発行ステーブルコインの可能性}}
\author{原 健太朗\footnote{Email: xharaken@gmail.com. 本稿に示された見解は個人のものであり私の所属する組織のものではありません。}}
\date{2022 September}

\begin{document}

\maketitle

\begin{abstract}

本稿では、仮に日銀がパブリックブロックチェーン上でステーブルコインを発行した場合、どのような可能性とリスクがあるかの思考実験を行う。具体的には、「日銀発行ステーブルコインによって日本円をデジタル通貨分野における基軸通貨に持ち上げることは可能か」という問題について考察する。

\end{abstract}

\section{基軸通貨と貨幣発行益}

日銀発行ステーブルコインの可能性を検討する準備として、基軸通貨発行国が持つ基軸通貨発行特権について確認する。

基軸通貨~\cite{monetarytheory,robert2004international}とは国際間の決済に広く用いられる通貨のことをいい、現在の基軸通貨は米ドルである。アメリカによって発行された米ドルはアメリカ国外で広く使用されており、たとえば日本とフィリピンが貿易する際の決済に米ドルが使用されたり、エクアドルやエルサルバドルのようなドル化された国々~\cite{calvo2002dollarization,selgin2005currency}で法定通貨として使用されている。

基軸通貨発行国であるアメリカは基軸通貨発行特権を持ち、膨大な貨幣発行益を手に入れている。日本が海外から必要商品を輸入したい場合には、代金支払いに必要な外貨を入手する必要があり、輸出または外貨借り入れによって外貨を調達する必要があるが、基軸通貨発行国であるアメリカは米ドルを刷るだけでよい。相手国が米ドルを受領するかぎりアメリカは米ドルを刷って輸入代金支払いに当てることができる。実際アメリカは恒常的な貿易赤字を抱えている。

この基軸通貨発行特権は非基軸通貨国に対して不公平を生み出す。たとえばドル化を採用しているエルサルバドルが100万米ドルを国内に流通させている状況を考える。この100万米ドルはどこから来たものかといえば、元をたどればエルサルバドルが100万米ドル分の輸出を(直接的または間接的にアメリカに対して)行って手に入れたものであるが、その100万米ドルはアメリカが刷ったものである。つまり100万米ドルの貨幣発行益はエルサルバドルではなくアメリカが手に入れている~\footnote{この不公平な状況を解消するには、エルサルバドルのような途上国が自国通貨主権を回復し、自国通貨の貨幣発行益を手に入れられるようにすることが必要である。ところが途上国は経済規模が小さく不安定なため、ドル化やドルペッグに頼ることなく自国通貨の通貨価値を安定させることが難しい。この問題を解決するために、著者はスマートコントラクトを利用したAlgorithmic Central Bankによって途上国が自国通貨主権を回復する手法を提案している~\cite{johnlawcoinacb}。}。

このように見るとアメリカが基軸通貨発行特権によって不当な利益を得ているように思われるが、アメリカは国際的な決済に広く用いられるのに十分なだけの米ドルを供給する責任を同時に負っていることも忘れてはならない。仮にアメリカが貿易赤字を解消するために輸入を過度に緊縮した場合、国際的な米ドルの流動性不足が発生して経済が混乱することになる。この点を指摘したのがトリフィンのジレンマである~\cite{bordo2019triffin}。トリフィンのジレンマの主張は、基軸通貨発行国は国際的な基軸通貨の流動性需要を満たすように基軸通貨を供給しなければならないが、そのとき基軸通貨発行国は貿易赤字を抱えることになり、貿易赤字は基軸通貨としての信認低下をもたらすというジレンマを抱えているというものである。ただし、著者の見解ではこれはジレンマではない。アメリカは米ドルの国際的な流動性需要を満たすように米ドルを供給しなければならず、実際に恒常的な貿易赤字を抱えているが、これは基軸通貨国として正常な状況であり、貿易赤字であるという事実のみによって基軸通貨としての信認低下が起きることはないと考えられる。基軸通貨国の貿易赤字(適切な流動性供給が行われているかぎり通貨の信認低下は起きない)と非基軸通貨国の貿易赤字(通貨の信認低下が起きうる)は本質的に別物であることに留意する必要がある。またマクロに見れば、基軸通貨国が貿易赤字であるということは、それ以外の非基軸通貨国は総合的に貿易黒字で外貨を蓄積しているということであり正常な状態と言える~\footnote{基軸通貨国が貿易黒字になることのほうが問題であると言える。}。

まとめると、基軸通貨発行国は国際的な決済に広く用いられるのに十分なだけの通貨を供給する責任を負っており、そのプロセスを通じて膨大な貨幣発行益を手に入れることができる。現在の基軸通貨は米ドルであり、通貨の使用には慣性の法則が働くことを考えれば、いったん成立した基軸通貨を別の通貨が置き換えるのは並大抵のことではない。一方でデジタル通貨の分野は、BitcoinやEthereumなどさまざまな非法定通貨が群雄割拠している状態であり、基軸通貨の地位を獲得した法定通貨はまだ存在しない。以下では、「デジタル通貨の分野で円が基軸通貨の地位を獲得することは可能かどうか」という問題を検討していく。

\section{日銀発行ステーブルコイン}

\subsection{アイディア}

まず日銀発行ステーブルコインの具体的なアイディアを示す。まず日銀はEthereumなどのパブリックブロックチェーン上にスマートコントラクトをデプロイして、ERC20トークン~\cite{erc20token}を発行する。発行ルールとしては、日銀は$X$円が預金された場合に$X$コインを発行し、$X$コインが返却された場合には$X$円との兌換を保証する。

イメージとしては日銀が法定通貨担保型ステーブルコイン~\cite{arner2020stablecoins,moin2020sok}を発行するようなものであるが、USDT~\cite{tether}やJPYCのような民間発行の法定通貨担保型ステーブルコインと日銀発行ステーブルコインとの間には決定的な違いが存在する。民間発行の法定通貨担保型ステーブルコインは兌換保証のために100\%の担保を抱えておく必要があり、したがって貨幣発行益はまったく発生せず~\footnote{兌換保証のため、担保を運用することも許されない。なおUSDTを発行するTether社は流通するUSDTに対して100\%担保を保有していないのではないかという疑惑がかけられている。}、手数料収入によってビジネスを成立させる必要がある。また発行元が100\%の担保保有を証明できない場合にステーブルコインの信認が失われるカウンタパーティリスクが存在する。これに対して日銀発行ステーブルコインは担保を保有する必要がない。なぜなら日銀は円を発行する権限を持っているからである。$X$円が預金されて$X$コインを発行する時点で、預金された$X$円は何かに使ってもよいし、円供給が増加することによるインフレや円安が心配であれば消せばよい。$X$コインの兌換が要求された時点では、日銀はつねに$X$円を刷って提供することができる。$X$円を刷ったり消したりする操作はコンピュータ上の数字を増減させるだけの操作であり、日銀のバランスシートを拡大・縮小するだけの操作である~\cite{mmt2016wray}。ある時点で日銀発行ステーブルコインが1000兆コイン流通している状況では、日銀は1000兆円の貨幣発行益を手に入れている。これは米ドルが国際的な決済手段として100兆米ドル流通している状況においてアメリカは基軸通貨国として100兆米ドルの貨幣発行益を手に入れていることに相当する。

日銀発行ステーブルコインはパブリックブロックチェーン上のERC20トークンとして発行されるため、世界中のユーザが非中央集権的にトランザクションを作成できる。米ドルを通じた海外送金と異なり、銀行口座を持つ必要はなく決済は少額手数料で即時に完了する。デジタル決済用途で利用するユーザにとっては、日銀ステーブルコインは既存のデジタル通貨(Bitcoin、Ethereum、USDT、JPYC、MakerDAO~\cite{makerdao}など)より魅力的に映ると考えられる。第一に、デジタル決済用途では貨幣価値が安定していることが重要であり、ステーブルコインはBitcoin、Ethereumなどの非ステーブルコインに対して優位性を持つ。第二に、USDTやJPYCなどの法定通貨担保型ステーブルコインと比較すると、それらが民間機関から発行されるコインであるのに対して日銀発行ステーブルコインは円の発行権限を持つ中央銀行によって発行されるコインでありカウンタパーティリスクが小さい。第三に、MakerDAOなどの暗号通貨担保型ステーブルコインは担保の暗号通貨の資産価値が暴落するリスクを抱えており、またガバナンストークンを利用したDAOで運用されるため将来どのようなルールでステーブルコインが発行管理されるかが予測できないリスクがあるのに対して、日銀発行ステーブルコインにはそのリスクが存在しない。第四に、Terra~\cite{terra}やEmpty Set Dollar~\cite{emptysetdollar}などのアルゴリズム型ステーブルコインは長期的に通貨価値を安定させることに失敗している~\footnote{著者の提案するJohnLawCoin~\cite{johnlawcoin}はアルゴリズム型ステーブルコインで長期的な通貨価値安定を目指すものである。}。以上の考察から、デジタル決済用途で利用するユーザにとって、日銀発行ステーブルコインは既存のデジタル通貨より優位性があると言える。

\subsection{リスク}

日銀発行ステーブルコインの最大のリスクはAML(Anti-Money Laundering)/CFT(Combating the Financing of Terrorism)~\cite{verdugo2011compliance}であると考えられる。AML/CFTに対しては技術面でも法制面でもさまざまな対応策が検討されているが、2019年に金融庁主導のもとで実施された「ブロックチェーンを用いた金融取引のプライバシー保護と追跡可能性に関する調査研究」~\cite{blockchainprivacy}に示されているように、ミキシング、ステルスアドレス、リング署名、ゼロ知識証明~\cite{sasson2014zerocash}などトランザクション匿名化の技術が多数開発されており、根本的にはパブリックブロックチェーン上でのトランザクション追跡は技術的に不可能と考えるべきである。したがって、日銀が日銀発行ステーブルコインを発行することは、マネーロンダリングやテロ資金供与の温床となりうるプラットフォームを日本が提供することを意味し、国際的な批判にさらされるリスクが存在する。

AML/CFTに関して、以下の3点を指摘しておきたい。

\begin{itemize}
\item AML/CFTは程度の問題でもあって、ステーブルコインを発行せずとも中央銀行が発行する紙幣はマネーロンダリングやテロ資金供与に利用されうる。
\item AML/CFTの対策としてはKYC(Know Your Customer)~\cite{parra2017kyc}が必要になるが、ある国の中央銀行が別の国の国民や企業に対して口座作成を許可したりKYCの情報を収集することは法的およびプライバシー的に非常に困難であると考えられる。Bank of Englandをはじめ各国の中央銀行が構想している中央銀行デジタル通貨(CBDC)~\cite{cbdc1,cbdc2}やエルサルバドルが提供するBitcoinウォレットであるChivoウォレットについてもKYCは自国内の国民や企業に制限されている。なお、この点こそが中央銀行デジタル通貨と日銀発行ステーブルコインとの差異でもある。中央銀行デジタル通貨は自国内の国民や企業に対してKYCを行い、中央銀行のプライベートなデータベースで運用される自国内決済システムであり、AML/CFT対策は施されているが、基軸通貨として国外で需要されることによる貨幣発行益は生じない。これに対して日銀発行ステーブルコインはパブリックブロックチェーン上のデジタル通貨であり、KYCを行わないためAML/CFT対策が不十分になるが基軸通貨として国外に需要されることによる貨幣発行益が存在する。
\item 将来的にデジタル通貨が広く利用されるようになった世界において、AML/CFTはトランザクション以外のレイヤで対策されるべき問題に変質する可能性がある。現時点でブロックチェーンにおけるAML/CFTが問題になるのは、「貨幣のトランザクションは追跡できるものである」という前提が人々の間に存在し、よって「マネーロンダリングやテロ資金供与のトランザクションは禁止できる」と人々が考えているからであるが、デジタル通貨の世界では匿名的なトランザクションを作成可能であり、この前提はすでに破綻している。追跡できないトランザクションが当然のように存在する世界になればこの前提が変化する可能性がある。たとえば、インターネットによってテロ組織が情報交換することは可能だが、インターネットの存在が問題視されることが少ないのは、それがもたらす便益の大きさと、インターネットが当然のものとなり追跡できない通信が存在することを人々が認めているからである。AML/CFTは、古典的な管理された銀行間取引に対して非中央集権的なデジタル通貨取引が出現した際の過渡期的現象とも捉えられる。ただし、仮にAML/CFTが過渡期的現象であるとしても現在はその過渡期にあるのだから現在AML/CFTが問題視される事実に変わりはない。
\end{itemize}

いずれにせよ、日銀発行ステーブルコインの最大のリスクはAML/CFTとどう向き合うかという点に存在する。

\subsection{貨幣発行益を手に入れることは望ましいことか}

本稿では日銀発行ステーブルコインによって日本円をデジタル通貨分野での基軸通貨に持ち上げそれによって貨幣発行益を手に入れることを構想してきたが、基軸通貨発行によって貨幣発行益を得ることが望ましいことかどうかについて考察しておきたい。

重要な事実は、日銀発行ステーブルコインを発行して1000兆円が流通した場合、それがなければ発生しえなかった1000兆円分の価値交換が生じているので、(マネーロンダリングやテロ資金供与などの望ましくない価値交換がないとすれば)大きな価値が生み出されているという点である。問題なのは、基軸通貨発行によって貨幣発行益を手に入れることではなく、その貨幣発行益を基軸通貨発行国がどう再分配するかである。日本が日銀発行ステーブルコインの貨幣発行益を国富として蓄えれば不当な利益を得ていると受け止められる可能性があるが、1000兆円の貨幣発行益を途上国支援に使うと約束すれば国際的な理解が得やすいものと思われる。仮に貨幣発行益の1000兆円全部を途上国支援に使ったとしても、日本の国際的なプレゼンスと円の競争力は高まることが予想され日本の国富にはつながることが期待される。このように日銀発行ステーブルコインの貨幣発行益の是非は、望ましい再分配の問題として議論されるべきである。

\section{結論}

本稿は日銀がパブリックブロックチェーン上でステーブルコインを発行した場合、どのような可能性とリスクがあるかについて考察した。著者の見解では、日銀発行ステーブルコインは日本円をデジタル通貨分野における基軸通貨に持ち上げる可能性を秘めており、AML/CFTとどう向き合うかがリスクであると考えられる。

本稿は日銀がステーブルコインを発行することを想定して記述してきたが、他国の中央銀行も同様の機会を有していることに留意したい。仮に中央銀行発行ステーブルコインが良いアイディアだとすれば、日銀が実行しないと他国が先に実行する可能性があり、通貨の使用には慣性の法則が働くことを考えれば先行した国ほど基軸通貨としての地位を獲得する可能性が高い。言い換えると、将来的に中央銀行発行ステーブルコインが成立しうるかどうかは国家戦略として検討に値する問題であると言える。


\bibliography{whitepaper.bib}
\bibliographystyle{unsrt}

\end{document}
